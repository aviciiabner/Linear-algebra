\documentclass[11pt]{ctexart}
\usepackage[margin=2cm,a4paper]{geometry}
\usepackage{amsthm, amsfonts, amsmath, amssymb, mathrsfs, newclude, tikz-cd, tikz, ctex, mathtools, stmaryrd, datetime}

%\setmainfont{Caladea}

%% 也可以选用其它字库:
% \setCJKmainfont[%
%   ItalicFont=AR PL KaitiM GB,
%   BoldFont=Noto Sans CJK SC,
% ]{Noto Serif CJK SC}
% \setCJKsansfont{Noto Sans CJK SC}
% \renewcommand{\kaishu}{\CJKfontspec{AR PL KaitiM GB}}



\usepackage[colorlinks = true,
linkcolor = blue,
urlcolor  = blue,
citecolor = blue,
anchorcolor = blue]{hyperref}

% Include the x-color package for color support
\usepackage{xcolor}

% Define a new environment for red comments
\usepackage{verbatim} % Required for the comment environment
\usepackage{environ}

\usepackage{mdframed} % Include mdframed for creating framed environments

\definecolor{pinked}{RGB}{255,231,229} % Define a base color 
% Define a new environment with a background color
\newmdenv[
  backgroundcolor=pinked, % Set the desired background color
  linecolor=white, % Optional: Set the border line color
  linewidth=1pt, % Optional: Set the border line width
  roundcorner=5pt, % Optional: Set rounded corners
  nobreak=true % Optional: Prevent page breaks within the environment
]{pinked}

\theoremstyle{definition}
\newtheorem{qqq}{问题}[section]

\newcommand{\ExternalLink}{%
    \tikz[x=1.2ex, y=1.2ex, baseline=-0.05ex]{% 
        \begin{scope}[x=1ex, y=1ex]
            \clip (-0.1,-0.1) 
                --++ (-0, 1.2) 
                --++ (0.6, 0) 
                --++ (0, -0.6) 
                --++ (0.6, 0) 
                --++ (0, -1);
            \path[draw, 
                line width = 0.5, 
                rounded corners=0.5] 
                (0,0) rectangle (1,1);
        \end{scope}
        \path[draw, line width = 0.5] (0.5, 0.5) 
            -- (1, 1);
        \path[draw, line width = 0.5] (0.6, 1) 
            -- (1, 1) -- (1, 0.6);
        }
    }

\NewEnviron{aaa}{~\\
    \noindent {\textcolor{teal}{\textbf{解答}} \BODY }
}

\NewEnviron{llll}{
    \noindent {~\\$\ExternalLink$ 外部链接 $\,\,\,$ \color{blue}\url{\BODY} }
}

\renewcommand{\proofname}{证明}
\renewcommand\qedsymbol{${\boxed{\substack{\textit{完证}\\\textit{毕明}}}}$}

% Change equation numbering to include the section number
\usepackage{cleveref}
\renewcommand{\theequation}{\thesection.\thesubsection.\arabic{equation}}
\numberwithin{equation}{section}

\usepackage{listings}
% Define listings style
\lstset{
  frame=tb,
  language=TeX,
  aboveskip=3mm,
  belowskip=3mm,
  showstringspaces=false,
  columns=flexible,
  basicstyle={\small\ttfamily},
  numbers=none,
  breaklines=true,
  breakatwhitespace=true,
  tabsize=3
}


\theoremstyle{definition}
\newtheorem*{definition}{定义}
\newtheorem*{proposition}{命题}
\newtheorem*{theorem}{定理}
\newtheorem*{notation}{记号}
\newtheorem*{example}{例子}
\newtheorem*{exercise}{习题}
\theoremstyle{remark}
\newtheorem*{remark}{备注}
\newtheorem*{lemma}{引理}
\newtheorem*{corollary}{推论}

\title{第二次作业反馈: \textbf{四元数不是域!}}
\author{(2024-2025-1)-MATH1405H-02}

\setcounter{page}{0}

\setlength\parindent{0pt}


\begin{document}

\maketitle

\tableofcontents

\newpage

\section{符号说明}

以下是几点符号说明. 

    \begin{enumerate}
        \item 向量直接写作 $v$, 不必使用粗体或上加箭头. 
        \item 我们不区分以下记号 $\subset$, $\subseteq$, $\subseteqq$与 $\sqsubset$, 但推荐使用前两者; 若想表示真包含, 请用 $\subsetneq$ 或 $\subsetneqq$. 
        \item 禁止使用 $\because$ 与 $\therefore$ (这是某种简便记号, 不应出现在正式写作中), 请直接用``因为''与``所以''. 
        \item 禁止使用 $\div$, 请使用分数或 $(-)^{-1}$. 
        \item 禁止使用 $C_n^p$ 表示组合数, 统一写作 $\binom np$. 
        \item 禁止使用 $\times$ 表示矩阵乘法, 请使用 $\cdot$ 或直接省略.  
        \item 禁止使用 $0$ 表示零矩阵 $O$; 但 $0$ 可以表示常数, 零向量, 以及零线性空间. 
    \end{enumerate}

\section{第二次作业 (自学任务)}

\subsection{第一个自学任务} 

\begin{exercise}
    结合以上定义, 
    \begin{enumerate}
        \item 若 $\varnothing_V$ 是 $V$ 的空子集, 则 $\mathrm{Span}_{(\mathbb F,V)}(\varnothing_V)$ 是什么? 
        \item 若 $S$ 是有限集, 请说明以上定义的 $\mathrm{Span}_{(\mathbb F,V)}(S)$ 与课堂中的表述一致. 
        \item 若 $S$ 是无限集, 则 $\mathrm{Span}_{(\mathbb F,V)}$ 是什么? 
    \end{enumerate}
    \begin{proof}
        如下. 
        \begin{enumerate}
            \item $\mathrm{Span}_{(\mathbb F,V)}(\varnothing_V)$ 是包含 $\varnothing$ 的最小的 $V$-线性子空间. 线性空间有 $0$, 容易检查 $\mathrm{Span}_{(\mathbb F,V)}(\varnothing_V)=0$. 
            \item 不妨设 $S=\{v_i\}_{i=1}^k$. 需要证明: 包含 $S$ 的最小线性子空间是
            \begin{equation}
                \left\{\sum_{i=1}^n c_i v_i\mid c_i\in \mathbb F\right\}.
            \end{equation}
            以上集合是线性子空间 (关于线性运算封闭), 且与 $\mathrm{Span}_{(\mathbb F,V)}(\varnothing _V)$ 互相包含. 
            \item 作为集合, $\mathrm{Span}_{(\mathbb F,V)}(S)=\bigcup_{S_0\subset S}\mathrm{Span}_{(\mathbb F,V)}(S_0)$, 其中 $S_0$ 取遍所有 $S$ 的有限子集. 
            \begin{pinked}
                注意: 线性空间没有 $\cup$ 这类运算, 但以上集合确实``自动是线性空间''. 由于检验线性空间``八条公理''的每一步骤都是对有限个对象进行的, 从而是在某一有限维线性子空间中进行的, 因此成立. 
            \end{pinked}
            此处的普适性原理: $\mathbb F$-线性空间向集合范畴的遗忘函子\href{https://ncatlab.org/nlab/show/created+limit}{生 (create)} 一切\href{https://ncatlab.org/nlab/show/inverse+limit}{极限 (inverse limit)} 与\href{https://ncatlab.org/nlab/show/filtered+colimit}{滤过余极限 (filtered co-limit)}, 类似的结论不胜列举. 简单地说, 
            \begin{pinked}
                (SLOGAN) 对线性空间考虑零点集, 无穷交, 以及建立等价关系等操作时, 只需在集合层面上操作, 得到的东西 (若存在) 自动是线性空间 (且唯一). 不需额外地进行验证.  
            \end{pinked}
            若无法理解, 那就认真写证明过程吧. 可以把这个 SLOGAN 看成 coincidence 而非 theorem. 
        \end{enumerate}
    \end{proof}
\end{exercise}

\subsection{习题课相关}

\begin{exercise}[1]
    State the definition of a number field, and prove that number fields are $\mathbb Q$-linear spaces. 
    \begin{proof}
        实际上, 但凡一个域包含 $\mathbb Q$ (严谨地说, 存在一个同构于 $\mathbb Q$ 的子域), 则该域是 $\mathbb Q$-线性空间. 直接验证即可. 借由此题, 我们将数域的定义统一如下: 
        \begin{pinked}
            \begin{definition}[数域]
                称 $K$ 是数域, 当且仅当 $K$ 是有限维 $\mathbb Q$-线性空间, 且 $K$ 是域. 
            \end{definition}
        \end{pinked}
    \end{proof}
\end{exercise}

\begin{remark}
    按照惯例, 今后作业 (或试题) 会在必要时标注
    \begin{pinked}
        Throughout, $\mathbb Q\subset \mathbb F\subset \mathbb C$. 
    \end{pinked}
以下是今后不多考虑的两类域. 二元域 (有限域) 必不是数域, 因为数域中 $(x\neq 0)\to (x+x\neq 0)$. 有理函数域 $\mathbb C(x)=\{p(x)/q(x)\}$ 看似比 $\mathbb C$ ``严格大'', 但在承认选择公理时仍能视作 $\mathbb C$ 的子域. 
\end{remark}

\begin{exercise}[2]
    Prove that the $3$-dimensional $\mathbb Q$-linear space
\begin{equation}
    V=\{a+b\cdot 2^{1/3}+c\cdot 2^{2/3}\mid a,b,c\in \mathbb Q\}
\end{equation}
 is a number field.  
 \begin{proof}
    不平凡的地方是构造分母有理化, 从而证明 $(a+b\sqrt[3]2+c\sqrt[3]4)^{-1}\in V$. 似乎可以在``谢书''上找到本题的原题 (解法也是暴力的分母有理化). ``聪明''的解法是证明有理化的存在性, 而非直接构造. 
    \begin{lemma}
        取 $x=a+b\sqrt[3]2+c\sqrt[3]4$, 存在一个次数不超过 $3$ 的非零多项式 $f$, 使得 $f(x)=0$. 
        \begin{proof}
            题中的记号 $V$ 引导我们将 $x$ 视作有理数的向量 $(a,b,c)\in \mathbb Q^3$. 此时, $x^2\in V$ 也有对应的有理数的向量形式. 由于 $V$ 是三维的, 从而四元集 $\{1,x,x^2,x^3\}$ 关于域 $\mathbb Q$ 线性相关. 考虑``表出其线性相关性''的系数, 得 $f$ 的系数. 
        \end{proof}
    \end{lemma}
    不妨设 $f$ 无法分解成真因子的乘积. 对 $x\neq 0$, 总有 $f(0)\neq 0$. 记 $f(t)=a_0+a_1t+a_2t^2+a_3t^3$, 那么
    \begin{equation}
        x^{-1}=\frac{a_1+a_2x^1+a_3x^2}{-a_0}\in V. 
    \end{equation}
 \end{proof}
\end{exercise}

\begin{remark}
    Challenge: generalise \& prove. 依照以上解法, 显然. 
\end{remark}

\begin{exercise}[3]
    Find a field $K$ such that $\mathbb C$ is a proper subfield of $K$. 
    \begin{proof}
        找到 $\mathbb C(x)$ 就足够了. \textbf{特别注意: 四元数不是域, 因为域的乘法需要交换!}
        \begin{itemize}
            \item 如果要求 ``$K$ 无法以任何形式视作 $\mathbb C$ 的子域'', 那就构造一个比 $\mathbb C$ 还大的集合 $S$ (例如 $\mathbb C$ 的幂集, 由理发师悖论). 最后构造 $\mathbb C$ 关于 $S$ 的自由扩张 (所有有限扩张的并集, 依照此文档 2.1.3 的``注意'', 这是一个域). 
        \end{itemize}
         
    \end{proof}
\end{exercise}

\begin{exercise}[4]
    Prove that $(1,e^x,e^{2x},\ldots, e^{2024 x})$ are linearly independent real-valued functions. 
    \begin{proof}
        (实值函数的) 积分和求导都是 (域 $\mathbb R$ 上的) 线性映射, 试在高中课本上寻找并补全相关定义. 
        \begin{pinked}
            对给定的线性映射 $T$, 则 $\{v_i\}_{i=1}^n$ 线性无关是 $\{T(v_i)\}_{i=1}^n$ 线性无关的必要条件. 
        \end{pinked}
        依逆否命题, 若存在非零系数的组合式使得 $\sum c_k e^{kx}=0$, 则对任意阶求导都有 $\sum c_k \frac{\operatorname{d}^n}{\operatorname{d} x^n} e^{kx}=0$. 换言之, 对 $n=0,1,2,\ldots, 2024$ 都有
        \begin{equation}
            \sum_{k=0}^{2024} k^n c_ke^{kx}=0\quad (n=0,1,\ldots, 2024). 
        \end{equation}
        采用矩阵表示, 即 
        \begin{equation}
            \begin{pmatrix}
                0^{0} & 1^{0} & 2^{0} & \cdots  & 2024^{0}\\
                0^{1} & 1^{1} & 2^{1} & \cdots  & 2024^{1}\\
                0^{2} & 1^{2} & 2^{2} & \cdots  & 2024^{2}\\
                \vdots  & \vdots  & \vdots  & \ddots  & \vdots \\
                0^{2024} & 1^{2024} & 2^{2024} & \cdots  & 2024^{2024}
                \end{pmatrix} \cdot \begin{pmatrix}
                c_{0} e^{0x}\\
                c_{1} e^{1x}\\
                c_{2} e^{2x}\\
                \vdots \\
                c_{2024} e^{2024x}
                \end{pmatrix} =\begin{pmatrix}
                0\\
                0\\
                0\\
                0\\
                0
                \end{pmatrix} .
        \end{equation}
        记上式为 $A\cdot b=0$. 依照 \href{https://en.wikipedia.org/wiki/Vandermonde_matrix}{Vandermonde matrix}, $A$ 可逆. 从而 $b=0$. 
    \end{proof}
\end{exercise}

\begin{exercise}[5]
    Find $n$ such that $\left(\sin \frac{\pi}{2n},\sin \frac{2\pi}{2n},\ldots, \sin \frac{(n-1)\pi}{2n}\right)$ are linearly dependent (over $\mathbb Q$). 
    \begin{proof}
        反例可以由``和差化积''公式以及 $2\cdot \cos\frac{\pi}{3}=1$ 轻易给出. 
    \end{proof}
\end{exercise}

\subsection{线性子空间}

\begin{example}[例题]
    证明以下两个句子描述了相同的子集. 
    \begin{enumerate}
        \item 既包含 $U_1$, 又包含 $U_2$ 的最小线性子空间.
        \begin{itemize}
            \item 需要说明存在性和唯一性.
        \end{itemize}
        \item  集合 $\left\{\sum u_1+u_2\mid u_1\in U_1, u_2\in U_2\right\}$.
        \begin{itemize}
            \item 无需顾虑空集, 因为线性空间必含有元素 $0$.
        \end{itemize}
        这一子集是线性空间, 记作 $U_1+U_2$. 
    \end{enumerate}
    \begin{proof}
        先证明存在包含 $U_1$ 且包含 $U_2$ 的最小线性子空间. 
        \begin{itemize}
            \item (存在性) 取子空间的集合 $\mathscr F=\{W\text{ 是全空间的子空间}\mid U_1\subset W, U_2\subset W\}$. 由于全空间属于 $\mathscr F$, 故 $\mathscr F$ 非空. 取以上所有子空间的交
            \begin{equation}
                W_0:=\bigcap _{W\in \mathscr F}W. 
            \end{equation}
            我们断言 $W_0$ 是线性空间 (从而是子空间). 
            \begin{itemize}
                \item[A] 对任意 $w,w'\in W_0$ 与常数 $\lambda$, 线性组合 $\lambda w+w'$ 属于所有 $W\in \mathscr F$, 从而属于 $W_0$. 
            \end{itemize}
            再断言 $W_0$ 是包含 $U_1$ 与 $U_2$ 的线性子空间. 证明类似上一断言. 
            \begin{itemize}
                \item[B] 选定任意 $u_i\in U_i$ ($i=1,2$). 由于 $u_i$ 属于一切 $W\in \mathscr F$, 则 $u_i$ 属于 $W_0$. 
            \end{itemize}
            最后断言 $W_0$ 是最小的线性子空间. 
            \begin{itemize}
                \item[C] 一切满足断言 A 与 B 的线性空间 $W'$ 必然属于 $\mathscr F$, 从而 $W_0\subset W'$. 
            \end{itemize}
            由 A, B 与 C, 存在性证毕 (即, 以上构造的 $W_0$ 是一解). 
            \item (唯一性) 若 $W_0'$ 也是``包含 $U_1$ 与 $U_2$ 的最小线性子空间'', 则 $W_0'\in \mathscr F$. 若同时再有 $W_0'\neq W_0$, 则必有 $W_0\subsetneq W_0'$. 这表明
            \begin{itemize}
                \item 若存在与 $W_0$ 不等的最小元 $W_0'$, 则 $W_0\subsetneq W_0'$ 是真包含关系. 矛盾. 
            \end{itemize}
            \begin{pinked}
                数学中区分``极小''与``最小''. 证明``最小元''的良定义性, 即证明``极小元''的唯一性. 
            \end{pinked}
            \item ($W_0$ 构造的集合一致) 由于 $U_1$ 与 $U_2$ 已经是线性空间了, 从而以上由有限和构造的集合就是
            \begin{equation}
                U_3:=\{u_1+u_2\mid u_1\in U_1,u_2\in U_2\}. 
            \end{equation}
            往后证明 $W_0=U_3$. 
            \begin{itemize}
                \item ($W_0\subset U_3$) 依照定义, $U_3\in \mathscr F$, 从而 $W_0\subset U_3$. 
                \item ($U_3\subset W_0$) 只需证明对任意 $W\in \mathscr F$ 都有 $U_3\subset W$. 由 $U_1\subset W$ 且 $U_2\subset W$, 则有 $(U_1\cup U_2)\subset W$. 由于 $W$ 对线性组合式封闭, 此时有 
                \begin{equation}
                    U_1+U_2=\operatorname{span}(U_1\cup U_2)\subset W.
                \end{equation}
            \end{itemize}
        \end{itemize}
    \end{proof}
\end{example}

\begin{example}[线性空间的``分配律'']
    观察子空间的式子 $(U_1\cap U_3)+(U_2\cap U_3)\subset (U_1+U_2)\cap U_3$. 
    \begin{itemize}
        \item (包含关系成立) 原理: 若 $A\subset B$ 与 $C\subset B$, 则有 $(A+C)\subset B$. 
        \item (不必取等) 以下所有 $U_i$ 都是 $\mathbb R^2$ 的一维子空间. 取 
        \begin{equation}
            U_1=\mathrm{Span}(\{(1,0)\}),\quad  U_2=\mathrm{Span}(\{(0,1)\}),\quad  U_3=\mathrm{Span}(\{(1,1)\}). 
        \end{equation} 
    \end{itemize}
\end{example}


\begin{example}[线性空间的``分配律'']
    观察子空间的式子 $(U_1+ U_3)\cap (U_2+ U_3)\supset (U_1\cap U_2)+ U_3$. 
    \begin{itemize}
        \item (反包含关系成立) 原理: 若 $A\supset B$ 与 $C\supset B$, 则有 $(A\cap C)\supset B$. 
        \item (不必取等) 以下所有 $U_i$ 都是 $\mathbb R^2$ 的一维子空间. 取 
        \begin{equation}
            U_1=\mathrm{Span}(\{(1,0)\}),\quad  U_2=\mathrm{Span}(\{(0,1)\}),\quad  U_3=\mathrm{Span}(\{(1,1)\}). 
        \end{equation} 
    \end{itemize}
\end{example}

\begin{remark}
    线性空间的全体子空间具有某种``前所未见的代数结构'': 对选定的 $\star\in \{\cap ,+\}$, 总有``零元'', 结合律, 以及交换律等; 但令人惊奇的是分配律不再成立. 刻画此类代数结构的``普适性框架''可以是\href{https://en.wikipedia.org/wiki/Lattice_(order)}{格}. 
\end{remark}

\begin{proposition}[子空间的模恒等式]
    对 $U_-\subset U_+$, 证明 $(U_-+V)\cap U_+\subset U_-+(V\cap U_+)$. 
    \begin{proof}
        一侧包含关系不依赖线性结构, 因此是普遍的. 下式中, $\text{任意字母}>\text{任意数字}$: 
        \begin{equation}
            \underset{A}{\underbracket{(U_-+V)}}\quad \cap\quad \underset{B}{\underbracket{U_+}}\quad \supset \quad \underset{1}{\underbracket{U_-}}\quad +\quad \underset{2}{\underbracket{(V\cap U_+)}}. 
        \end{equation}
        另一侧包含关系证明如下. 任取 $x\in (U_-+V)\cap U_+$, 则存在 $a\in U_-$ 与 $b\in V$ 使得 $x=a+b$. 
        \begin{itemize}
            \item 断言 $(x-a)\in V$, 因为 $(x-a)=b\in V$. 
            \item 断言 $(x-a)\in U_+$, 因为 $x\in U_+$ 且 $a\in U_+$.
        \end{itemize}
因此 $x=a+(x-a)\in U_- + (V\cap U_+)$. 
    \end{proof}
\end{proposition}

\begin{remark}
    线性空间具有\href{https://en.wikipedia.org/wiki/Modular_lattice}{模格结构}, 即某种二维的保守场. 今后的线性同态基本定理, 子空间对应定理, Noether 同构定理, Zassenhaus 引理, 以及谱序列的计算均基于此. 这一性质对全体加群, 模等 (良幂的) Abel 范畴均成立. 
\end{remark}

\begin{remark}
    本次思考题比较宽泛, 就不统一给出``答案''了. 感兴趣的同学可以自行询问.
\end{remark}

\section{\textbf{重要: 四元数不是域!}}

\begin{pinked}
    \textbf{重要: 四元数不是域! 域的乘法需要是交换的!} 
\end{pinked}

\end{document}