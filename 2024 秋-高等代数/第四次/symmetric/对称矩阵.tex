\documentclass[11pt]{ctexart}
\usepackage[margin=2cm,a4paper]{geometry}
\usepackage{amsthm, amsfonts, amsmath, amssymb, mathrsfs, newclude, tikz-cd, tikz, ctex, mathtools, stmaryrd, datetime}

%\setmainfont{Caladea}

%% 也可以选用其它字库:
% \setCJKmainfont[%
%   ItalicFont=AR PL KaitiM GB,
%   BoldFont=Noto Sans CJK SC,
% ]{Noto Serif CJK SC}
% \setCJKsansfont{Noto Sans CJK SC}
% \renewcommand{\kaishu}{\CJKfontspec{AR PL KaitiM GB}}



\usepackage[colorlinks = true,
linkcolor = blue,
urlcolor  = blue,
citecolor = blue,
anchorcolor = blue]{hyperref}

% Include the x-color package for color support
\usepackage{xcolor}

% Define a new environment for red comments
\usepackage{verbatim} % Required for the comment environment
\usepackage{environ}

\usepackage{mdframed} % Include mdframed for creating framed environments

\definecolor{pinked}{RGB}{255,231,229} % Define a base color 
% Define a new environment with a background color
\newmdenv[
  backgroundcolor=pinked, % Set the desired background color
  linecolor=white, % Optional: Set the border line color
  linewidth=1pt, % Optional: Set the border line width
  roundcorner=5pt, % Optional: Set rounded corners
  nobreak=true % Optional: Prevent page breaks within the environment
]{pinked}

\theoremstyle{definition}
\newtheorem{qqq}{问题}[section]

\newcommand{\ExternalLink}{%
    \tikz[x=1.2ex, y=1.2ex, baseline=-0.05ex]{% 
        \begin{scope}[x=1ex, y=1ex]
            \clip (-0.1,-0.1) 
                --++ (-0, 1.2) 
                --++ (0.6, 0) 
                --++ (0, -0.6) 
                --++ (0.6, 0) 
                --++ (0, -1);
            \path[draw, 
                line width = 0.5, 
                rounded corners=0.5] 
                (0,0) rectangle (1,1);
        \end{scope}
        \path[draw, line width = 0.5] (0.5, 0.5) 
            -- (1, 1);
        \path[draw, line width = 0.5] (0.6, 1) 
            -- (1, 1) -- (1, 0.6);
        }
    }

\NewEnviron{aaa}{~\\
    \noindent {\textcolor{teal}{\textbf{解答}} \BODY }
}

\NewEnviron{llll}{
    \noindent {~\\$\ExternalLink$ 外部链接 $\,\,\,$ \color{blue}\url{\BODY} }
}

\renewcommand{\proofname}{证明}
\renewcommand\qedsymbol{${\boxed{\substack{\textit{完证}\\\textit{毕明}}}}$}
\let\oldproof\proof
\renewcommand{\proof}{\color{blue}\oldproof}

% Change equation numbering to include the section number
\usepackage{cleveref}
\renewcommand{\theequation}{\thesection.\thesubsection.\arabic{equation}}
\numberwithin{equation}{section}

\usepackage{listings}
% Define listings style
\lstset{
  frame=tb,
  language=TeX,
  aboveskip=3mm,
  belowskip=3mm,
  showstringspaces=false,
  columns=flexible,
  basicstyle={\small\ttfamily},
  numbers=none,
  breaklines=true,
  breakatwhitespace=true,
  tabsize=3
}

\theoremstyle{definition}
\newtheorem*{definition}{定义}
\newtheorem*{proposition}{命题}
\newtheorem*{theorem}{定理}
\newtheorem*{notation}{记号}
\newtheorem*{example}{例子}
\newtheorem{exercise}{\textcolor{blue}{习题}}
\theoremstyle{remark}
\newtheorem*{remark}{备注}
\newtheorem*{lemma}{引理}
\newtheorem*{corollary}{推论}

\title{习题: 对称矩阵 (\LaTeX 重排)}
\author{(2024-2025-1)-MATH1405H-02}

\setcounter{page}{0}

\setlength\parindent{0pt}

\begin{document}

\maketitle

\vspace{5cm}

\section*{前言}

\textbf{只需完成习题 1 与习题 2}. 若想提升个人实力, 建议完成剩下的题目. 

\begin{itemize}
    \item 习题 3 与习题 4 的唯一作用是提升计算能力, 同时熟悉一些特殊矩阵的结构.
    \item 习题 5 是脑经急转弯, 请勿擅自点拨他人.
    \item 习题 6 是去年的考试题, 可以挑战在 15 分钟内完成.   
\end{itemize}

\begin{pinked}
\LaTeX 重排版修改了部分题目以及提示. 并未改动习题 1 或习题 2. 
\end{pinked}

\newpage

\section*{习题内容}

\begin{exercise}
对称矩阵的判断题. 不必证明真命题, 但需要对假命题举出反例.
\begin{enumerate}
    \item 线性空间 $\mathbb F^{n\times n}$ 中的对称矩阵构成 $\binom{n+1}{2}=\frac{n^2+n}{2}$-维子空间. 
    \begin{proof}
        正确. 基底可取 $\{E_{i,i}\}_{i=1}^n\cup \{E_{i,j}+E_{j,i}\}_{1\leq i<j\leq n}$. 
    \end{proof}
    \item 分块矩阵 $\begin{pmatrix}A&B\\C&D\end{pmatrix}$ 是对称的, 当且仅当 $B=C^T$. 
    \begin{proof}
        错误. $A$ 与 $D$ 不必是对称矩阵. 
    \end{proof}
    \item 给定阶数相同的对称方阵 $A$ 与 $B$, 则 $A\cdot B$ 亦对称. 
    \begin{proof}
        错误. 例如 $\binom{0\,\,1}{1\,\,0}\cdot \binom{1\,\,0}{0\,\,0}=\binom{0\,\,1}{0\,\,0}$. 
    \end{proof}
    \item 给定阶数相同的方阵 $A$ 与 $B$, 若 $A\cdot B=B\cdot A$ 对称, 则 $A$ 与 $B$ 必有一者对称.
    \begin{proof}
        错误. 例如 $A$ 不对称, $B=O$. 
    \end{proof}
    \item 若 $A^2$ 是对称矩阵, 则 $A$ 对称. 
    \begin{proof}
        错误. 例如 $A=\binom{0\,\,1}{0\,\,0}$. 
    \end{proof}
\end{enumerate} 
\end{exercise}

\begin{exercise}
    若 $X$ 与线性空间 $\mathbb F^{n\times n}$ 中一切对称矩阵乘积可交换, 试求 $X$. 
    \begin{proof}
        假定 $A$ 与所有对称矩阵交换. $A$ 的存在性是显然的, 例如 $A=O$. 
        \begin{itemize}
            \item $A$ 是对角矩阵. 对所有形如 $E_{i,i}$ 的矩阵, 均有 $E_{i,i}A=AE_{i,i}$, 此时 $A$ 对角. 
            \item $A$ 是数乘矩阵. 记 $A=\sum c_i E_{i,i}$, 由 $A(E_{i,j}+E_{j,i})=(E_{i,j}+E_{j,i})A$ 可知 $c_i=c_j$. 
        \end{itemize}
    \end{proof}
\end{exercise}

\begin{exercise}
    给出以下方程的一个解: 
\begin{equation}
    L\cdot L^T=\begin{pmatrix}3&2&1\\2&2&1\\1&1&1\end{pmatrix}. 
\end{equation}
其中, 要求 $L$ 为下三角矩阵
\begin{equation}
    \begin{pmatrix}a&\\b&c\\d&e&f\end{pmatrix}. 
\end{equation}
\begin{itemize}
    \item 以上是数值分析中常见的 Cholesky 分解. 这一个分解对所有 (半) 正定矩阵奏效.
\end{itemize} 
\begin{proof}
    直接地, $a^2=9$. 取 $a=\sqrt 3$, 则得 $L$ 的第一纵列
    \begin{equation}
        L=\begin{pmatrix}\sqrt 3&\\{2}/{\sqrt 3}&c\\{1}/{\sqrt 3}&e&f\end{pmatrix}. 
    \end{equation}
    此时 $\frac 43+c^2=2$. 不妨取 $c=\frac{\sqrt 2}{\sqrt 3}$, 则得 $L$ 的第二纵列
    \begin{equation}
        L=\begin{pmatrix}\sqrt 3&\\{2}/{\sqrt 3}&{\sqrt 2}/{\sqrt 3}\\{1}/{\sqrt 3}&{\sqrt 2}/{2\sqrt 3}&{\sqrt 2}/{2}\end{pmatrix}=\begin{pmatrix}\sqrt 3&\\\frac{2\sqrt 3}{3}&\frac{\sqrt 6}{3}\\\frac{\sqrt 3}{3}&\frac{\sqrt 6}{6}&\frac{\sqrt 2}{2}\end{pmatrix}. 
    \end{equation}
\end{proof}
\end{exercise}

\begin{exercise}
    给定 $(a_1,a_2,a_3,a_4)$, 求解以下方程中的 $(x_1,x_2,x_3)$. 
\begin{equation}
    \begin{pmatrix}
        &&&1\\&&1&x_1\\&1&x_1&x_2\\1&x_1&x_2&x_3
        \end{pmatrix}
        \cdot 
        \begin{pmatrix}
        &&&a_4\\1&&&a_3\\&1&&a_2\\&&1&a_1
        \end{pmatrix}
        =
        \begin{pmatrix}
        &1&&\\&&1&\\&&&1\\a_4&a_3&a_2&a_1
        \end{pmatrix}
        \cdot 
        \begin{pmatrix}
        &&&1\\&&1&x_1\\&1&x_1&x_2\\1&x_1&x_2&x_3
        \end{pmatrix}.
\end{equation}
空缺位置的元素都是 $0$. 
\begin{itemize}
    \item 结合有理标准型, 以上构造间接解答了以下问题: 任意域上的方阵 $A$ 通过对称矩阵与其转置相似, 即, 存在对称矩阵 $S$ 使得 $S^{-1}AS=A^T$.
\end{itemize}
\begin{proof}
    直接地, 上述方程有解, 当且仅当以下矩阵是对称的
    \begin{equation}
        \begin{pmatrix}
            0 & 0 & 1 & a_{1}\\
            0 & 1 & x_{1} & a_{2} +x_{1} a_{1}\\
            1 & x_{1} & x_{2} & a_{3} +x_{1} a_{2} +x_{2} a_{1}\\
            x_{1} & x_{2} & x_{3} & a_{4} +x_{1} a_{3} +x_{2} a_{2} +x_{3} a_{1}
            \end{pmatrix}.
    \end{equation}
    解得
    \begin{equation}
        x_1=a_1,\quad x_2=a_2+a_1^2,\quad x_3=a_3+2a_1a_2+a_1^3. 
    \end{equation}
\end{proof}
\end{exercise}

\begin{exercise}
    (这是一道脑经急转弯, 无需计算) 将以下两个实矩阵分解作 $2$ 个实对称矩阵的乘积. 
\begin{equation}
    \begin{pmatrix}
        \lambda&1\\&\lambda &1\\&&\lambda
        \end{pmatrix},\quad \begin{pmatrix}
        a&b\\-b&a&1\\&&a&b\\&&-b&a\\
        \end{pmatrix}. 
\end{equation}
\begin{itemize}
    \item 依照复矩阵的 Jordan 型与实矩阵的旋转-反射标准型, 任意实方阵 (相应地, 复方阵) 一定是两个实对称矩阵 (相应地, 复对称矩阵) 的乘积. 
    \item 作为推论, 两个对称矩阵的乘积不必对称. 
\end{itemize}
\begin{proof}
    考虑矩阵 $\begin{pmatrix}
        &&1\\&1&\\1&&
        \end{pmatrix}$. 
\end{proof}
\end{exercise}

\begin{exercise}
    (去年的一道期中考题, 大约占分 $15/100$) 若实方阵 $A$ 满足 $A(A-A^T)=O$, 证明 $A=A^T$. 
    \begin{proof}
        若 $A^2=AA^T$, 则 $\mathrm{tr}(A^2)=\mathrm{tr}(AA^T)$. 展开得 
        \begin{equation}
            \sum_{i,j=1}^n a_{i,j}a_{j,i}=\sum_{i,j=1}^n a_{i,j}^2. 
        \end{equation}
        配凑完全平方式, 得 $\sum_{i,j=1}^n(a_{i,j}-a_{j,i})^2=0$. 此时 $A=A^T$. 
    \end{proof}
    \begin{itemize}
        \item 若认为考试题比较简单, 可以尝试由 $A(A-A^T)A=O$ 推导 $A=A^T$. 此处可以借用实矩阵的正交标准型: 
\begin{equation}
    A=Q^T\cdot \begin{pmatrix} S&R\\O&O\\\end{pmatrix}\cdot Q.
\end{equation}
        其中, $(S\quad R)$ 是行满秩矩阵, $S$ 是方阵. \textcolor{red}{$S$ 未必可逆, 反例如 $\begin{pmatrix}0 & 1\\0 & 0 \end{pmatrix}$}\footnote{QR 分解只能保证行满秩, ``$S$ 可逆''仅对本题等特殊情况成立. 这一反例由王子涵同学提供.}. 
    \end{itemize}
    \begin{proof}
        假定 $A^3=AA^TA$. 对 $A^3A^T=AA^TAA^T$ 使用上述标准型, 得
        \begin{equation}
            \begin{pmatrix} S^3S^T+S^2RR^T&O\\O&O\\\end{pmatrix}=\begin{pmatrix} (SS^T+RR^T)^2&O\\O&O\\\end{pmatrix}. 
        \end{equation}
        比较分块矩阵的左上处得 $S^2(SS^T+RR^T)=(SS^T+RR^T)^2$. 由于 $(S\quad R)$ 行满秩, 
        \begin{equation}
            \det (SS^T+RR^T)=\det \left(\begin{pmatrix} S\quad R\end{pmatrix}\cdot \begin{pmatrix} S\quad R\end{pmatrix}^T\right) >0. 
        \end{equation}
        比较分块矩阵左上角的行列式, 得 $(\det S)^2\cdot \det (SS^T+RR^T)=\det (SS^T+RR^T)^2$, 即
        \begin{equation}
            \det (SS^T)=\det (SS^T+RR^T).
        \end{equation}
        依照 Cauchy-Binet 公式, 只能有 $R=O$. 依照行满秩条件, 此时 $S$ 可逆, 故 $S=S^T$. 
    \end{proof}
\end{exercise}







\end{document}