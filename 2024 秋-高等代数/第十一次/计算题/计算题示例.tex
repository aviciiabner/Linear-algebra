\documentclass[12pt, reqno]{amsart}
\usepackage{amsthm, graphicx, color, tikz-cd, mathrsfs, stackrel, float}
\usepackage[bookmarksnumbered, colorlinks, plainpages]{hyperref}

\usepackage{fontspec}

\usepackage{unicode-math} % Modern font handling (requires XeLaTeX or LuaLaTeX)

\usepackage{amsthm}

\fontsize{7}{8}\selectfont

% \setmathfont{XITS Math}

\usepackage{parskip}


\textheight 22.5truecm \textwidth 14.5truecm
\setlength{\oddsidemargin}{0.35in}\setlength{\evensidemargin}{0.35in}

\setlength{\topmargin}{-.5cm}

\theoremstyle{plain}

\newtheorem{theorem}{Theorem}[section]
\newtheorem{lemma}[theorem]{Lemma}
\newtheorem{proposition}[theorem]{Proposition}
\newtheorem{corollary}[theorem]{Corollary}
\theoremstyle{definition}
\newtheorem{definition}[theorem]{Definition}
\newtheorem{example}[theorem]{Example}
\newtheorem*{exercise}{Exercise}
\newtheorem{conclusion}[theorem]{Conclusion}
\newtheorem{conjecture}[theorem]{Conjecture}
\newtheorem{criterion}[theorem]{Criterion}
\newtheorem{summary}[theorem]{Summary}
\newtheorem{axiom}[theorem]{Axiom}
\newtheorem{problem}[theorem]{Problem}
\theoremstyle{remark}
\newtheorem{remark}[theorem]{Remark}
\numberwithin{equation}{section}

\newtheorem*{solution}{Solution}

\setlength\parindent{0pt} % Set Noindent for the Entire Document. 

\begin{document}
\setcounter{page}{1}


\noindent {\small MATH 1405(H), Qiang Ji.}\hfill  {\small }\\
{\small \today}\hfill  {\small }

\centerline{}

\centerline{}

\title{Calculation Problems in Linear Algebra}

\author{Chencheng Zhang}

%In case of 3 or more authors use below format
%\author[F. Author, S. Author, T. Author]{First Author$^1$, Second Author$^2$$^{*}$ \MakeLowercase {and} Third Author$^3$}

\address{School of mathematics, Shanghai Jiao Tong University, Shanghai, PRC.}
\email{\textcolor[rgb]{0.00,0.00,0.84}{zhangchencheng@sjtu.edu.cn}}

\makeatletter
\@namedef{subjclassname@2020}{\textup{2020} Mathematics Subject Classification}
\makeatother

%\dedicatory{This paper is dedicated to Professor ABCD}

\subjclass[2020]{Primary 20J06}

\keywords{linear algebra, linear system, matrix decompositions}

\begin{abstract}
    This document enumerates the types of calculation questions that may appear in the examination, serving as a reference for identifying potential knowledge gaps.

    The document is (partly) written in Unicode-math.
\end{abstract} \maketitle








\section{Linear Systems}

\begin{example}\label{Quadratic forms}
    Let
    \begin{equation}
        x_1^2+x_2^2+5x_3^2+2tx_1x_2-2x_1x_3+4x_2x_3
    \end{equation}
    be a real quadratic form. Please give the value of $t$ so that $f(x)$ is positive definite.
\end{example}

\begin{example}\label{echelon}
    Find the reduced row échelon form of the following matrix,
    \begin{equation}
        \begin{pmatrix}
            0 & 3  & −6 & 6  & 4 & −5 \\
            3 & −7 & 8  & −5 & 8 & 9  \\
            3 & −9 & 12 & −9 & 6 & 15
        \end{pmatrix}.
    \end{equation}
    and prove that the reduced row échelon form is always unique.
\end{example}




\section{Linear Maps}






\section{Matrix Calculations}

\subsection{Determinants}

\begin{example}\label{det_quick_glance}
    Find the coefficient of $x^3$ in the following determinant
    \begin{equation}
        f(x) = \det  \begin{pmatrix}
            4x & 3x & 2  & 1  \\
            1  & x  & 1  & -1 \\
            3  & 2  & 2x & 1  \\
            1  & 0  & 1  & x
        \end{pmatrix}.
    \end{equation}
\end{example}

\begin{example}\label{methodologies}
    Summarise the techniques in HW7 (\href{https://oc.sjtu.edu.cn/courses/72790/assignments/312156?module_item_id=1294252}{Canvas Page}).
\end{example}





\section{Matrix Decompositions}

\subsection{Polar, \texorpdfstring{$ℝ$}{PDFstring}-symmetric}

\begin{example}\label{Cholesky}
    Find a $3×3$ real upper triangular matrix $T$ with positive diagonal entries, such that
    \begin{equation}
        T^T ⋅ T =
        \begin{pmatrix}
            16 & 8  & -4 \\
            8  & 5  & -4 \\
            -4 & -4 & 14
        \end{pmatrix}.
    \end{equation}
    It is also worthwhile to prove the uniqueness of $T$.
\end{example}






\section{Answers}

\begin{solution}\ref{Quadratic forms}
    $(-∞, -⅘) ∪ (0, +∞)$.
\end{solution}

\begin{solution}\ref{Cholesky}
    The desired matrix is
    \begin{equation}
        \begin{pmatrix}
            4 & 2 & -1 \\0&1&2\\0&0&3
        \end{pmatrix}.
    \end{equation}
    The uniqueness: if there is another $L′$, then there exists some orthogonal matrix $Q$ such that $L = Q ⋅ L′$. Since $Q = L ⋅ (L′)⁻¹$ is upper triangular with all positive eigenvalues, $Q = I$.
    \begin{remark}
        What if $T^T ⋅ T$ is not a square matrix? To what extent the decomposition is unique? Hint: a priori examination.
    \end{remark}
\end{solution}

\begin{solution}\ref{det_quick_glance}
    $-6$. One should recognise it ``efficiently''.
\end{solution}

\begin{solution}\ref{methodologies}
    Some feasible techniques:
    \begin{enumerate}
        \item Inductions.
        \item Eigenvalues.
        \item $λ^m ⋅ \det (λ I_n - AB) = λ^n ⋅ \det (λ I_m - BA)$, wherein the corollary
              \begin{equation}
                  \det(I-AB) = \det(I-BA)
              \end{equation}
              is more used.
        \item add-one-line trick: $\det \begin{pmatrix}
                      1 & ∗ \\ 0 & A
                  \end{pmatrix} = \det A$.
        \item Schur complement. when $\det\begin{pmatrix}
                      A & B \\C&D
                  \end{pmatrix}=\det(AD-BC)$?
        \item Laplace expansion, cofactor expansion, and neglect the linearly dependent sets when necessary:
              \begin{enumerate}
                  \item for instance, $\det (A + u ⋅ v^T) = \det A + ∑\limits_{\text{$n$-terms}} + ∑ \limits_\text{$2^n - n - 1$-zeros}$;
                  \item another example: $\det (\cos (i⋅ θ_j)) = \det (2^{1-i}⋅ \cos^i (θ_j))$.
              \end{enumerate}
        \item Cauchy-Binet formula. An distinguished example:
              \begin{equation}
                  \det (R^TR + S^TS) = \det (R\quad S)^T ⋅ (R\quad S) \overset{\text{CB}}{ = }  \cdots ≥ (\det R)^2,
              \end{equation}
              equality holds whence $S=O$.
        \item By factors (e.g., Bi-Vandermonde determinant).
        \item ...
    \end{enumerate}
\end{solution}

\begin{solution}\ref{echelon}
    Ans:
    \begin{equation}
        \begin{pmatrix}
            1 & 0 & −2 & 3 & 0 & −24 \\
            0 & 1 & −2 & 2 & 0 & −7  \\
            0 & 0 & 0  & 0 & 1 & 4
        \end{pmatrix}.
    \end{equation}
    See HW2 for the proof of uniqueness.
\end{solution}



\end{document}


