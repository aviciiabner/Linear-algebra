% Options for packages loaded elsewhere
\PassOptionsToPackage{unicode}{hyperref}
\PassOptionsToPackage{hyphens}{url}
\documentclass[
]{ctexart}
\usepackage{xcolor}
\usepackage{amsmath,amssymb}
\setcounter{secnumdepth}{-\maxdimen} % remove section numbering
\usepackage{iftex}
\ifPDFTeX
  \usepackage[T1]{fontenc}
  \usepackage[utf8]{inputenc}
  \usepackage{textcomp} % provide euro and other symbols
\else % if luatex or xetex
  \usepackage{unicode-math} % this also loads fontspec
  \defaultfontfeatures{Scale=MatchLowercase}
  \defaultfontfeatures[\rmfamily]{Ligatures=TeX,Scale=1}
\fi
\usepackage{lmodern}
\ifPDFTeX\else
  % xetex/luatex font selection
\fi
% Use upquote if available, for straight quotes in verbatim environments
\IfFileExists{upquote.sty}{\usepackage{upquote}}{}
\IfFileExists{microtype.sty}{% use microtype if available
  \usepackage[]{microtype}
  \UseMicrotypeSet[protrusion]{basicmath} % disable protrusion for tt fonts
}{}
\makeatletter
\@ifundefined{KOMAClassName}{% if non-KOMA class
  \IfFileExists{parskip.sty}{%
    \usepackage{parskip}
  }{% else
    \setlength{\parindent}{0pt}
    \setlength{\parskip}{6pt plus 2pt minus 1pt}}
}{% if KOMA class
  \KOMAoptions{parskip=half}}
\makeatother
\setlength{\emergencystretch}{3em} % prevent overfull lines
\providecommand{\tightlist}{%
  \setlength{\itemsep}{0pt}\setlength{\parskip}{0pt}}
\usepackage{bookmark}
\IfFileExists{xurl.sty}{\usepackage{xurl}}{} % add URL line breaks if available
\urlstyle{same}
\hypersetup{
  hidelinks,
  pdfcreator={LaTeX via pandoc}}

\author{}
\date{}

\begin{document}

\textbf{Ex 1.} 直接写出以下矩阵的行列式, 或简要说明其行列式的求解方式.

\begin{quote}
\(\lambda\in \mathbb F\) 是给定的常数, \(A\in \mathbb F^{n\times n}\)
是矩阵.
\end{quote}

\begin{enumerate}
\def\labelenumi{\arabic{enumi}.}
\item
  置换矩阵.

  \begin{quote}
  答: \((-1)^{\text{逆序数}}\).
  \end{quote}
\item
  初等变换矩阵 \(D^\lambda_i\), \(T^\lambda_{j,i}\), 以及 \(S_{i,j}\).

  \begin{quote}
  答: \(\det D^\lambda_i=\lambda\), \(\det T^\lambda_{j,i}=1\), 以及
  \(\det S_{i,j}=-1\).
  \end{quote}
\item
  若 \(A\) 是对角矩阵, 求 \(\det A\).

  \begin{quote}
  答: 对角元乘积.
  \end{quote}
\item
  若 \(A\) 是上三角矩阵, 求 \(\det A\).

  \begin{quote}
  答: 对角元乘积.
  \end{quote}
\item
  若 \(A=\begin{pmatrix}X&O\\Y&Z\end{pmatrix}\), 其中 \(X\) 与 \(Z\)
  是方阵, 求 \(\det A\).

  \begin{quote}
  答: \(\det A =\det X\cdot \det Z\).
  \end{quote}
\item
  \(A^{-1}\) (若存在) 的行列式.

  \begin{quote}
  答: \(\det (A^{-1})=(\det A)^{-1}\).
  \end{quote}
\item
  方阵乘积的行列式.

  \begin{quote}
  答: \(\det AB=\det A\cdot \det B\).
  \end{quote}
\item
  若 \(\mathrm{rank}(A)<n\), 求 \(\det A\).

  \begin{quote}
  答: \(0\).
  \end{quote}
\item
  \(\lambda A\) 的行列式 (用 \(\det A\) 表示).

  \begin{quote}
  答: \(\lambda^{\text{矩阵阶数}}\cdot \det A\).
  \end{quote}
\item
  \(A^T\) 的行列式 (用 \(\det A\) 表示).

  \begin{quote}
  答: \(\det A^T=\det A\).
  \end{quote}
\item
  将 \(A\) 顺时针旋转 \(\pi/2\) 后的行列式 (用 \(\det A\) 表示).

  \begin{quote}
  答: 顺时针旋转 \(\pi/2\) 后取转置, 无非对换第 \(i\) 行与第 \(n-i\) 行
  (取遍 \(1\leq i\leq \lfloor n/2\rfloor\)). 符号 \((-1)^{n(n-1)/2}\).
  \end{quote}
\item
  \(f\) 是 \(\mathbb F\) 上的多项式, 求 \(\det (f(A))\).

  \begin{quote}
  答: 没什么确切的答案. 错误答案: \(f(\det A)\).
  \end{quote}
\item
  求 \(\det e^A\).

  \begin{quote}
  答: (为定义 \(e^A\), 需默认数域). 依照 \(e^{x+y}=e^x\cdot e^y\),
  可以猜到答案是 \(e^{\mathrm{tr}(A)}\). 严格的证明步骤如下:

  \begin{enumerate}
  \def\labelenumii{\arabic{enumii}.}
  \item
    在 \(\mathbb C\) 上使用 Jordan 标准型 \(A=P^{-1}JP\). 依照 \(e\)
    的级数定义得

    \[e^{P^{-1}JP}=P^{-1}e^JP\]
  \item
    由于 \(J\) 是上三角矩阵, 故 \(e^J\) 上三角, 且 \(e^J\) 在 \((i,i)\)
    处分量是 \(J\) 在 \((i,i)\) 处分量的指数. 因此
    \(\det e^J=e^{\mathrm{tr}(J)}\) 从而
    \(\det e^A=e^{\mathrm{tr}(A)}\).
  \item
    依照定义, \(\mathrm{tr}\) 与 \(\det\) 不依赖域的选取. 式
    \(\det e^A=e^{\mathrm{tr}(A)}\) 在 \(\mathbb F\) 上成立, 当且仅当在
    \(\mathbb C\) 上成立.
  \end{enumerate}
  \end{quote}
\end{enumerate}

\textbf{Ex 2.} 试比较以下.

\begin{enumerate}
\def\labelenumi{\arabic{enumi}.}
\item
  举出 \(\det (A-B)=0\) 但 \(\det (A^2-B^2)=1\) 的例子.

  \begin{quote}
  答: 例如 \(A=\begin{pmatrix}0&1\\1&0\end{pmatrix}\) 与
  \(B=\begin{pmatrix}0&1\\0&0\end{pmatrix}\).
  \end{quote}
\item
  举出 \(\det (A-B)=1\) 但 \(\det (A^2-B^2)=0\) 的例子.

  \begin{quote}
  答: 例如 \(A=(1/2)\) 与 \(B=(-1/2)\).
  \end{quote}
\item
  假设 \(AB=BA\), 则 \(\det (A^2-B^2)=\det (A-B)\det (A+B)\).

  \begin{quote}
  答: 交换性条件保证了 \(A^2-B^2=(A-B)(A+B)\).
  \end{quote}
\end{enumerate}

记 \(M=\begin{pmatrix}A&B\\C&D\end{pmatrix}\), 其中
\(A,B,C,D\in \mathbb R^{2\times 2}\). 记 \(N=DA-CB\).

\begin{quote}
此处的 \(\mathbb R\) 可以换做一般的域.
\end{quote}

\begin{enumerate}
\def\labelenumi{\arabic{enumi}.}
\item
  举出 \(\det M=0\) 但 \(\det N\neq 0\) 的例子.

  \begin{quote}
  答: 找出 \(A=C\), \(B=D\), 以及 \(DA-AD\) 可逆的例子即可.

  例如, 考虑 \(A=C=\begin{pmatrix}1&1\\0&1\end{pmatrix}\) 与
  \(B=D=\begin{pmatrix}1&0\\1&1\end{pmatrix}\). 由 \(r(M)=2\) 知
  \(\det M=0\). 另一方面,
  \(N=DA-AD=\begin{pmatrix}1&0\\0&-1\end{pmatrix}\) 可逆.
  这一构造对任意域也是成立的, 因为 \(-1\) 一定是域的可逆元.
  \end{quote}
\item
  举出 \(\det M\neq 0\) 但 \(\det N=0\) 的例子.

  \begin{quote}
  答: 依照下一问的提示, 每对相邻的 \(2\times 2\) 方块乘法不可交换.
  简单试验得

  \[M=\begin{pmatrix}1&0&0&1\\0&0&1&0\\0&1&1&0\\1&0&0&0\end{pmatrix}.\]

  经初等变换, \(M\) 可以变作置换矩阵, 从而 \(\det M\neq 0\). 计算知
  \(\det N=0\). 这一构造对任意域也是成立的.
  \end{quote}
\item
  假设 \(AB=BA\), 则 \(\det M = \det N\). 对称的命题略.

  \begin{quote}
  答: 以下的说理方式在一般域上可行, 但需要一些多项式,
  扩域方面的知识储备. 我们仅考虑 \(A,B,C,D\in \mathbb F^{n\times n}\)
  (\(\mathbb F\) 是数域).

  若 \(AB=BA\), 则有以下分块矩阵的恒等式

  \[\begin{pmatrix}A&B\\C&D\end{pmatrix}\cdot \begin{pmatrix}O&B\\I&-A\end{pmatrix}=\begin{pmatrix}B&O\\D&CB-DA\end{pmatrix}.\]

  因此 \(\det M\cdot \det B\cdot (-1)^{n^2}=\det B\cdot \det (CB-DA)\).
  依照 \(\det (CB-DA)=(-1)^{n}\det (DA-CB)\), 得

  \[\det B\cdot (\det M-\det (DA-CB))=0.\]

  对任意 \(x\in \mathbb F\), 记 \(B^{(x)}:= B+xI\), 以及
  \(M^{(x)}:=\begin{pmatrix}A&B^{(x)}\\C&D\end{pmatrix}\). 此时仍有
  \(AB^{(x)}=B^{(x)}A\). 遂有

  \[\det B^{(x)}\cdot (\det M^{(x)}-\det (DA-CB^{(x)}))=0.\]

  \begin{itemize}
  \item
    试回顾这一个事实: 若数域上的两个多项式满足 \(f\cdot g=0\), 则
    \(f=0\) 或 \(g=0\). 若将数域换成一般域, 需要做一些细微调整.
  \end{itemize}

  依照行列式的展开式 (或考虑特征值), 当 \(x\) 足够大时
  \(\det B^{(x)}\neq 0\). 这说明

  \[\det M^{(x)}-\det (DA-CB^{(x)})\]

  是恒零的多项式. 取 \(x=0\), 得证.
  \end{quote}
\item
  计算 \(\det \begin{pmatrix}I&B\\C&D\end{pmatrix}\).

  \begin{quote}
  答: 使用上一问的结论, 得 \(\det (D-CB)\).
  \end{quote}
\end{enumerate}

\textbf{Ex 3.} 使用矩阵初等变换, 证明对任意
\(A\in \mathbb F^{m\times n}\), \(B\in \mathbb F^{n\times m}\), 以及
\(\lambda\in \mathbb F\), 总有

\[\lambda^n\cdot \det (\lambda I_m-AB)=\lambda^m\cdot \det (\lambda I_n-BA).\]

\begin{quote}
答: 对 \(\lambda\neq 0\), 考虑列初等变换

\[\begin{pmatrix}\lambda I_m-AB&O\\B&\lambda I_n\end{pmatrix}\overset{\text{行变换}}\sim \begin{pmatrix}\lambda I_m&\lambda A\\B&\lambda I_n\end{pmatrix}\overset{\text{列变换}}\sim  \begin{pmatrix}\lambda I_m&O\\B&\lambda I_n- BA\end{pmatrix}.\]

从而

\[\lambda^n\cdot \det (\lambda I_m-AB)=\lambda^m\cdot \det (\lambda I_n-BA).\]
\end{quote}

\begin{itemize}
\item
  推广: 对方阵 \(A\), \(B\) 与 \(C\) (未必可逆), 总有
  \(\det (A+B+ACB)=\det (A+B+BCA)\).
\end{itemize}

\begin{quote}
答: 类似 Ex2-3 的证明. 选定数域, 取 \(A^{(x)}:=A+xI\) 以及
\(B^{(x)}:=B+xI\). 对足够大的 \(x\), \(A^{(x)}\) 与 \(B^{(x)}\)
均是可逆的. 此时

\begin{align}
\det(A^{(x)}+B^{(x)}+A^{(x)}CB^{(x)})&=\det A^{(x)}\cdot \det B^{(x)}\cdot \det ((A^{(x)})^{-1}+(B^{(x)})^{-1}+C),\\[6pt]
\det(A^{(x)}+B^{(x)}+B^{(x)}CA^{(x)})&=\det A^{(x)}\cdot \det B^{(x)}\cdot \det ((A^{(x)})^{-1}+(B^{(x)})^{-1}+C).
\end{align}

因此
\(\det(A^{(x)}+B^{(x)}+A^{(x)}CB^{(x)})-\det(A^{(x)}+B^{(x)}+B^{(x)}CA^{(x)})\)
是恒零的多项式. 取 \(x=0\) 即可.

\begin{itemize}
\item
  此题似乎用 Ex. 2-4 的结论就行了; 但 Ex. 2-4 本质上还是涉及了扰动法.
\end{itemize}
\end{quote}

 

\textbf{Ex 4.} 求以下矩阵行列式

\[\begin{pmatrix}
0 &  &  &  & a_{n}\\
1 & 0 &  &  & a_{n-1}\\
 & 1 & \ddots  &  & \vdots \\
 &  & \ddots  & 0 & a_{2}\\
 &  &  & 1 & a_{1}
\end{pmatrix}.\]

\begin{quote}
答: 此题过于简单了. 将最后一列提至第一列, 得上三角矩阵. 符号
\((-1)^{n-1}\cdot a_n\).

原题是求解特征多项式 \(\det (xI-A)\).

\[\det (xI-A)=-(a_n+a_{n-1}x+\cdots + a_1x^{n-1})+x^n.\]
\end{quote}

\textbf{Ex 5.} 以下是三对角矩阵的行列式问题.

\begin{quote}
注: 这是习题课的原题. 证明题的解答从略.
\end{quote}

\begin{enumerate}
\def\labelenumi{\arabic{enumi}.}
\item
  求以下三对角矩阵的行列式

  \[\begin{pmatrix}
  a & b &  &  & \\
  c & a & b &  & \\
  & c & \ddots  & \ddots  & \\
  &  & \ddots  & a & b\\
  &  &  & c & a
  \end{pmatrix}.\]

  提示: 使用归纳法, 需讨论 \(a^2=4bc\) 与否.

  \begin{quote}
  答: 记 \(D_n\) 是 \(n\) 阶此形式矩阵的行列式, 则有递推式

  \[D_1=a,\quad D_2= a^2-bc,\quad D_{n+2}=aD_{n+1}-bc D_n .\]

  \begin{itemize}
  \item
    今假定在合适的扩域下, \(x^2-ax+bc=0\) 的两解是 \(x_1\) 与 \(x_2\).
    或简单地说, 假定矩阵在数域上, 则方程在 \(\mathbb C\) 上的根是
    \(x_1\) 与 \(x_2\).
  \end{itemize}

  若 \(a^2\neq 4bc\), 则 \(D_n = \frac{x_1^{n+1}-x_2^{n-1}}{x_1-x_2}\).

  若 \(a^2=4bc\), 则 \(D_n=(n+1)\cdot (a/2)^n\).
  \end{quote}
\item
  原题有误, 现已删去. 应当是左侧行列式与右侧分子相同.
\item
  证明

  \[\det\begin{pmatrix}
  a_{1} & b_{1} &  &  & \\
  c_{1} & a_{2} & b_{2} &  & \\
  & c_{2} & \ddots  & \ddots  & \\
  &  & \ddots  & a_{n-1} & b_{n-1}\\
  &  &  & c_{n-1} & a_{n}
  \end{pmatrix} =\begin{pmatrix}
  a_{1} & b_{1}
  \end{pmatrix}\begin{pmatrix}
  a_{2} & b_{2}\\
  -c_{1} & 0
  \end{pmatrix} \cdots \begin{pmatrix}
  a_{n-1} & b_{n-1}\\
  -c_{n-2} & 0
  \end{pmatrix}\begin{pmatrix}
  a_{n}\\
  -c_{n-1}
  \end{pmatrix}\]

  \begin{quote}
  特别注明: 可以采用此结论证明第一问.
  \(\begin{pmatrix}a&b\\-c&0\end{pmatrix}^k\) 的计算方式是通常是对角化,
  这也回到了高中所学的所谓特征根.
  \end{quote}
\end{enumerate}

\textbf{Ex 6.} Vandermonde 矩阵的行列式.

\begin{enumerate}
\def\labelenumi{\arabic{enumi}.}
\item
  记 \(V:=(x_i^j)\in \mathbb F^{n\times n}\), 直接写出 \(\det V\).

  \begin{quote}
  答: 这与通常的 Vandermonde 矩阵稍有不同, 行列式是
  \(x_1\cdots x_n\cdot \prod_{1\leq i<j\leq n}(x_j-x_i)\).
  \end{quote}
\item
  将 \(V\) 删去 \(k\) 行与 \(k\) 列, 得 \(V'\). 求 \(\det V'\).

  \begin{quote}
  答: 此方法可以用于求解 Vandermonde 矩阵的 \(\mathrm{Adj}\)-矩阵,
  从而求得其逆矩阵.

  \begin{itemize}
  \item
    为表述方便, 我们假定将矩阵删去第 \(k\) 列 (所有 \(k\)-次幂) 与第
    \(1\) 行, 并计算新矩阵的行列式.
  \end{itemize}

  在原矩阵中记 \(t=x_1\) 与 \(y_k=x_{k+1}\) (\(1\leq k \leq n-1\)),
  则行列式为

  \[(-1)^{n-1}(y_1\cdots y_{n-1}\cdot t)\prod_{1\leq i<j\leq n-1}(y_j-y_i)\cdot \prod_{1\leq l\leq n-1}(t-y_l).\]

  新矩阵的行列式即 \(t^k\)-项系数的 \((-1)^{k-1}\) 倍.
  \end{quote}
\item
  将 \(V\) 的各项 (共 \(n^2\) 项) 加上 \(1\), 求新矩阵的行列式.

  \begin{quote}
  答: 使用加边技巧
  \(\det A=\det \begin{pmatrix}1&\\\alpha &A\end{pmatrix}\). 此处取
  \(A\) 为新矩阵, \(\alpha=\mathbf 1\). 使用初等变换,

  \[\det A = \det \begin{pmatrix}1&-\mathbf 1^T\\\mathbf 1 & V\end{pmatrix}.\]

  行列式关于第一行向量是线性的, 从而原式是两个 Vandermonde 行列式差.

  \[\det \begin{pmatrix}1&-\mathbf 1^T\\\mathbf 1 & V\end{pmatrix}=\det \begin{pmatrix}2&0\\\mathbf 1 & V\end{pmatrix}-\det \begin{pmatrix}1&\mathbf 1^T\\\mathbf 1 & V\end{pmatrix}.\]

  计算得

  \[\prod_{1\leq i<j\leq n}(x_j-x_i)\cdot \left(2\prod_{l=1}^nx_l-\prod_{l=1}^n (x_l-1)\right).\]
  \end{quote}
\item
  记 \(\{x_i\}_{i=1}^n\) 是整数, 证明
  \(\prod_{1\leq i<j\leq n}\frac{x_i-x_j}{i-j}\) 是整数.

  \begin{quote}
  提示: 记 \(\binom{n}{k}=C_n^k\) 为组合数. 假定所有 \(x_i\) 充分大,
  考虑 \(\det(\binom{x_i}{j})\).
  \end{quote}

  \begin{quote}
  答: \(\{\binom{n}{k}\}_{k\geq 0}\) 作为 \(n\) 的多项式是线性无关组,
  因此可以只看各组合数的最高次项. 使用 Vandermonde 行列式计算得

  \[\det\left (\binom{x_i}{j}\right ) = \prod_{1\leq i<j\leq n}\frac{x_i-x_j}{i-j}.\]

  由于这是整数矩阵的行列式, 从而是整数.
  \end{quote}
\end{enumerate}

\textbf{Ex 7.} 令
\(P=\left(\begin{array}{cccc}a_1&a_2&\cdots& a_n\\b_1&b_2&\cdots& b_n\\\end{array}\right)\),
\(Q=\left(\begin{array}{cccc}c_1&c_2&\cdots& c_n\\d_1&d_2&\cdots& d_n\\\end{array}\right)\).
对 \(\det (PQ^T)\) 使用 Cauchy-Binet 公式,
并与直接计算行列式所得的结果比较, 得 Lagrange 恒等式 (请验证):

\[\sum_{i=1}^n(a_ic_i)\sum_{i=1}^n(b_id_i)=\sum_{i=1}^n(a_id_i)\sum_{i=1}^n(b_ic_i)+\sum_{1\leq i_1<i_2\leq n}(a_{i_1}b_{i_2}-a_{i_2}b_{i_1})(c_{i_1}d_{i_2}-c_{i_2}d_{i_1}).\]

特别地, 对向量 \(\mathbf a, \mathbf b\in \mathbb R^n\), 证明

\[\|\mathbf a\|^2\|\mathbf b\|^2=(\mathbf a\cdot\mathbf b)^2+\sum_{1\leq i_1<i_2\leq n}(a_{i_1}b_{i_2}-a_{i_2}b_{i_1})^2.\]

若 \(n=3\), 试求 \(\|\mathbf a\times \mathbf b\|\)?

\begin{quote}
答: 一方面, Cauchy-Binet 公式给出

\[\det (PQ^T)=\sum_{1\leq i_1<i_2\leq n}\det\begin{pmatrix}a_{i_1}&b_{i_1}\\a_{i_2}&b_{i_2}\end{pmatrix}\cdot \det\begin{pmatrix}c_{i_1}&d_{i_1}\\c_{i_2}&d_{i_2}\end{pmatrix};\]

另一方面, 直接计算得

\[\det (PQ^T)=\sum_{i=1}^n(a_ic_i)\sum_{i=1}^n(b_id_i)-\sum_{i=1}^n(a_id_i)\sum_{i=1}^n(b_ic_i).\]

后略.
\end{quote}

\textbf{Ex 8.} 取 \((a_i)_{i\geq 1}\) 是周期为 \(n\) 的 \(\mathbb F\)
中的数列, 定义 \(n\times n\) 矩阵的第 \((i,j)\) 项为 \(a_{i+j-1}\).
计算这一循环矩阵的行列式.

\begin{quote}
答: 不妨将矩阵翻转 (符号改变方式见 Ex.1-11), 记矩阵 \(A\) 的第 \((i,j)\)
项为 \(a_{n+1+i-j}\).

记 \(\Omega\) 的 \((i,j)\) 项是 \(\omega^{(i-1)(j-1)}\), 其中
\(\omega=e^{2\pi i /n}\) 是单位根. 记多项式

\[f(x)=a_1+a_2x+a_3x^2+\cdots+a_nx^{n-1}.\]

将 \(A\) 左乘在 \(\Omega\) 的第 \(k\) 列上, 相当于数乘
\(f(\omega^{k-1})\). 因此

\[\det A\cdot \det \Omega =\det (A\Omega)=f(1)f(\omega)\cdots f(\omega^{n-1})\cdot \det \Omega.\]

依照 Vandermonde 矩阵可逆, \(\det\Omega\neq 0\). 从而
\(\det A= \prod _{0\leq l\leq n-1}f(\omega^l)\).
\end{quote}

\textbf{Ex 9.} 给定常数 \((c_1,\ldots, c_n)\). 试计算
\((c_{\min (i,j)})\in \mathbb F^{n\times n}\) 的行列式.

\begin{quote}
答: \(c_1\cdot (c_2-c_1)\cdot (c_3-c_2)\cdots (c_n-c_{n-1})\).
\end{quote}

\textbf{Ex 10.} 计算 Hilbert 矩阵的行列式. 关于 Hilbert 矩阵的定义,
以及此题答案可参考逆矩阵习题.

\begin{quote}
答: 答案见先前习题:

\[\det \left(\frac{1}{x _i+y _j}\right) _{1\le i,j \le n}=\frac{\prod _{1\le i < j\le n} (x _j-x _i)(y _j-y _i)}{\prod _{i,j=1}^n (x _i+y _j)}.\]
\end{quote}

\textbf{Ex 11.} 计算并总结以下行列式的通式

\[\det \begin{pmatrix}
1 & -1 & 0 & 0 & 0\\
x & h & -1 & 0 & 0\\
x^{2} & hx & h & -1 & 0\\
x^{3} & hx^{2} & hx & h & -1\\
x^{4} & hx^{3} & hx^{2} & hx & h
\end{pmatrix}.\]

\begin{quote}
答: 直接归纳得 \((x+h)^4\).
\end{quote}

\textbf{Ex 12.} 记分块矩阵
\(A=\begin{pmatrix}A_1&A_2\\A_3&A_4\end{pmatrix}\) 与
\(B=\begin{pmatrix}B_1&B_2\\B_3&B_4\end{pmatrix}\), 满足 \(r(A)=r(A_1)\)
与 \(r(B)=r(B_1)\). 此时

\[\det (A+B) \cdot \det (A_1)\cdot \det (B_4)=\det \begin{pmatrix}A_1&A_2\\B_3&B_4\end{pmatrix}\cdot \begin{pmatrix}A_1&B_2\\A_3&B_4\end{pmatrix}.\]

\begin{quote}
答: 若将题设改作 \(r(B)=r(B_4)\), 解答会清晰许多.

依题意, 存在矩阵 \(P,Q,R,S\) 使得

\[A=\begin{pmatrix}A_1&A_1P\\QA_1&QA_1P\end{pmatrix},\quad B=\begin{pmatrix}SB_4R&SB_4\\B_4R&B_4\end{pmatrix}.\]

此时

\begin{align*}
\det A_1 B_4\cdot \det (A+B)&=\det A_1B_4\cdot \det \begin{pmatrix}A_1+SB_4R&A_1P+SB_4\\QA_1+B_4R&QA_1P+B_4\end{pmatrix}\\[6pt]
&=\det A_1B_4\cdot  \det \begin{pmatrix}I&S\\Q&I\end{pmatrix}\cdot \begin{pmatrix}A_1&A_1P\\B_4R&B_4\end{pmatrix}\\[6pt]
&=\det \begin{pmatrix}A_1&SB_4\\QA_1&B_4\end{pmatrix}\cdot \begin{pmatrix}A_1&A_1P\\B_4R&B_4\end{pmatrix}
\end{align*}.

原题的解答类似, 或将上述 \(\{S,R\}\) 扰动作满秩矩阵.
\end{quote}

\textbf{Ex 13.} (Ptolemy 定理) 给定矩阵
\(\begin{pmatrix}a_1&a_2&a_3&a_4\\b_1&b_2&b_3&b_4\end{pmatrix}\), 记
\(\Delta_{i,j}=\det \begin{pmatrix}a_i&a_j\\b_i&b_j\end{pmatrix}\). 证明

\[\Delta_{1,2}\Delta_{3,4}+\Delta_{1,4}\Delta_{2,3}=\Delta_{1,3}\Delta_{2,4}.\]

\begin{quote}
答: 略.
\end{quote}

\textbf{Ex 14.} 通常的正整数矩阵 \(A\in \mathbb F^{n\times n}\)
的行列式. 其中 \(a_{i,j}=\gcd (i,j)\) 是最小公倍数.

提示: 证明对任意整数都有 \(m=\sum_{k\mid m}\phi (k)\). 其中 \(k\) 取遍
\(m\) 的所有因子, \(\phi\) 是通常的
\href{https://en.wikipedia.org/wiki/Euler\%27s_totient_function}{Euler
totient 函数}. 此时
\(a_{i,j}=\sum_{k\mid i \text{ 且 }k \mid j}\phi (k)\). 这表明
\(A=X^T\cdot D\cdot X\), 其中

\begin{itemize}
\item
  \(D=\mathrm{diag}(\phi(1),\phi(2),\ldots, \phi (n))\) 是对角矩阵,
\item
  \(X\) 是 \(\{0,1\}\)-下三角矩阵, 其中 \(X_{i,j}=1\) 当且仅当
  \(j\mid i\).
\end{itemize}

\begin{quote}
答: 提示已经很详细了. 答案是 \(\det D=\prod_{1\leq i\leq n}\phi (i)\).
\end{quote}

\textbf{Ex 15.} (对任意域而言) 若 \(A^T=-A\), 则 \(\det A\)
是完全平方式. 这称作 Pfaffian.

\begin{itemize}
\item
  特殊的 Pfaffian (来自 Cauchy 矩阵的基本性质):
  \(\det \left(\frac{x_i-x_j}{x_i+x_j}\right)_{n\times n}=\left(\prod_{i<j}\frac{x_i-x_j}{x_i+x_j}\right)^2\).
\end{itemize}

\begin{quote}
答: 假定结论对 \(n\)-阶矩阵成立, 往证结论对 \(n+2\) 阶反对称矩阵 \(A\)
成立. 可以对 \(A\) 做相同的行交换与列交换 (不改变行列式), 使得新矩阵的
\((n+1,n+2)\)-项不为零. 因此不妨设 \(A\) 的 \((n+1,n+2)\)-项不为零.
考虑分块矩阵

\[A=\begin{pmatrix}A' &B\\B^T&U\end{pmatrix},\quad U=\begin{pmatrix}0&x\\-x&0\end{pmatrix}.\]

依照 Schur 补, \(\det A = \det (A'-BU^{-1}B^T)\cdot \det U\).
结合归纳假设, \(\det A\) 是完全平方式.

\begin{itemize}
\item
  可以证明, 完全平方式 \(\det A = p^2\) 中, \(p\) 是以 \(\{a_{i,j}\}\)
  为系数的. 将 \(p\) 写作 \(x\) 的有理函数, 需证明所有
  \(x^{-k}\)-项系数为 \(0\). 依照行列式的定义, \(\det A\) 中不出现
  \(x^{-k}\) 之类的项.
\end{itemize}
\end{quote}

\textbf{Ex 16.} 记 \(a,b,c\) 是常数. 若矩阵 \(A\) 的严格下三角部分均为
\(a\), 严格上三角部分均为 \(b\), 对角线上均为 \(c\). 求 \(\det A\).

一般地, 记多项式 \(f(x)=c_0+c_1x+c_2x^2+\cdots c_{n-1}x^{n-1}\), 考虑
\(n\) 阶 \(\mathbb C\)-方阵

\[M:=\begin{pmatrix}
c_{0} & c_{1} & c_{2} & \cdots  & c_{n-2} & c_{n-1}\\
zc_{n-1} & c_{0} & c_{1} & \cdots  & c_{n-3} & c_{n-2}\\
zc_{n-2} & zc_{n-1} & c_{0} & \cdots  & c_{n-4} & c_{n-3}\\
\vdots  & \vdots  & \vdots  & \ddots  & \vdots  & \vdots \\
zc_{2} & zc_{3} & zc_{4} & \cdots  & c_{0} & c_{1}\\
zc_{1} & zc_{2} & zc_{3} & \cdots  & zc_{n-1} & c_{0}
\end{pmatrix}.\]

则 \(\det M=\prod_{k=1}^n f(w_k)\), 其中 \(w_k\) 是 \(w^n=z\) 的 \(n\)
个复根.

\begin{quote}
答: 见 Ex. 8.
\end{quote}

\textbf{Ex 17.} 记 \((a_i)_{i=1}^n\) 与 \((b_i)_{i=1}^n\) 是给定的常数,
且 \(a_ib_j\neq 1\). 记 \(m_{i,j}=\frac{1-(a_ib_j)^n}{1-a_ib_j}\), 计算
\(\det (m_{i,j})\).

\begin{quote}
答: 可以直接归纳计算, 或是使用如下技巧:

\begin{itemize}
\item
  对任意 \(t:=a_ib_j\) 行列式是关于 \(t\) 的 \(n-1\) 次多项式;
\item
  行列式以一切 \(a_i-a_j\) 与 \(b_i-b_j\) 为因子.
\end{itemize}

从而行列式只能是 \(\prod_{1\leq i<j\leq n}(a_i-a_j)(b_i-b_j)\) 的数乘倍.
检验 \(a_1b_1\) 的系数, 以上就是行列式的值.
\end{quote}

\begin{quote}
注: 以上行列式是两个 Vandermonde 行列式的乘积,
的确可以使用矩阵乘积来计算 \(M\). 此外, 计算 \(m_{i,j}=(a_i+b_j)^n\)
的行列式时, 也会出现类似双 Vandermonde 行列式 (差一个数乘倍).
\end{quote}

\textbf{Ex 18.} 假定域的特征不为 \(2\). 证明: 对任意方阵 \(A\),
总存在一个取值 \(\{\pm 1\}\) 的对角矩阵 \(D\) 使得 \(\det (A+D)\neq 0\).

\begin{quote}
答: 使用数学归纳. \(n=1\) 显然; 假定 \(n=k\) 成立, 下证 \(n=k+1\) 成立.
任取 \(k+1\)-阶矩阵 \(M\), 记分块矩阵

\[M=\begin{pmatrix}A&v\\ u^T&\lambda\end{pmatrix}.\]

存在 \(D'\) 使得 \((D'+A)\) 可逆, 下证
\(M+\begin{pmatrix}D'&\\&1\end{pmatrix}\) 或
\(M+\begin{pmatrix}D'&\\&-1\end{pmatrix}\) 可逆即可.
由于行列式关于最后一列线性, 故

\begin{align}
&\quad \ \ \det \left(M+\begin{pmatrix}D'&\\&1\end{pmatrix}\right)-\det \left(M+\begin{pmatrix}D'&\\&-1\end{pmatrix}\right)\\[6pt]
&=\det \begin{pmatrix}A+D'&v\\ u^T&\lambda+1\end{pmatrix}-\det \begin{pmatrix}A+D'&v\\ u^T&\lambda-1\end{pmatrix}\\[6pt]
&=\det \begin{pmatrix}A+D'&v\\ &2\end{pmatrix}\quad \neq 0.
\end{align}
\end{quote}

\end{document}
