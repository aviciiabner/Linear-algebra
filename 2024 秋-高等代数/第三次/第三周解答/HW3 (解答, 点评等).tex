\documentclass[11pt]{ctexart}
\usepackage[margin=2cm,a4paper]{geometry}
\usepackage{amsthm, amsfonts, amsmath, amssymb, mathrsfs, newclude, tikz-cd, tikz, ctex, mathtools, stmaryrd, datetime}

%\setmainfont{Caladea}

%% 也可以选用其它字库:
% \setCJKmainfont[%
%   ItalicFont=AR PL KaitiM GB,
%   BoldFont=Noto Sans CJK SC,
% ]{Noto Serif CJK SC}
% \setCJKsansfont{Noto Sans CJK SC}
% \renewcommand{\kaishu}{\CJKfontspec{AR PL KaitiM GB}}



\usepackage[colorlinks = true,
linkcolor = blue,
urlcolor  = blue,
citecolor = blue,
anchorcolor = blue]{hyperref}

% Include the x-color package for color support
\usepackage{xcolor}

% Define a new environment for red comments
\usepackage{verbatim} % Required for the comment environment
\usepackage{environ}

\usepackage{mdframed} % Include mdframed for creating framed environments

\definecolor{pinked}{RGB}{255,231,229} % Define a base color 
% Define a new environment with a background color
\newmdenv[
  backgroundcolor=pinked, % Set the desired background color
  linecolor=white, % Optional: Set the border line color
  linewidth=1pt, % Optional: Set the border line width
  roundcorner=5pt, % Optional: Set rounded corners
  nobreak=true % Optional: Prevent page breaks within the environment
]{pinked}

\theoremstyle{definition}
\newtheorem{qqq}{问题}[section]

\newcommand{\ExternalLink}{%
    \tikz[x=1.2ex, y=1.2ex, baseline=-0.05ex]{% 
        \begin{scope}[x=1ex, y=1ex]
            \clip (-0.1,-0.1) 
                --++ (-0, 1.2) 
                --++ (0.6, 0) 
                --++ (0, -0.6) 
                --++ (0.6, 0) 
                --++ (0, -1);
            \path[draw, 
                line width = 0.5, 
                rounded corners=0.5] 
                (0,0) rectangle (1,1);
        \end{scope}
        \path[draw, line width = 0.5] (0.5, 0.5) 
            -- (1, 1);
        \path[draw, line width = 0.5] (0.6, 1) 
            -- (1, 1) -- (1, 0.6);
        }
    }

\NewEnviron{aaa}{~\\
    \noindent {\textcolor{teal}{\textbf{解答}} \BODY }
}

\NewEnviron{llll}{
    \noindent {~\\$\ExternalLink$ 外部链接 $\,\,\,$ \color{blue}\url{\BODY} }
}

\renewcommand{\proofname}{证明}
\renewcommand\qedsymbol{${\boxed{\substack{\textit{完证}\\\textit{毕明}}}}$}

% Change equation numbering to include the section number
\usepackage{cleveref}
\renewcommand{\theequation}{\thesection.\thesubsection.\arabic{equation}}
\numberwithin{equation}{section}

\usepackage{listings}
% Define listings style
\lstset{
  frame=tb,
  language=TeX,
  aboveskip=3mm,
  belowskip=3mm,
  showstringspaces=false,
  columns=flexible,
  basicstyle={\small\ttfamily},
  numbers=none,
  breaklines=true,
  breakatwhitespace=true,
  tabsize=3
}


\theoremstyle{definition}
\newtheorem*{definition}{定义}
\newtheorem*{proposition}{命题}
\newtheorem*{theorem}{定理}
\newtheorem*{notation}{记号}
\newtheorem*{example}{例子}
\newtheorem*{exercise}{习题}
\theoremstyle{remark}
\newtheorem*{remark}{备注}
\newtheorem*{lemma}{引理}
\newtheorem*{corollary}{推论}

\title{第三次作业反馈}
\author{(2024-2025-1)-MATH1405H-02}

\setcounter{page}{0}

\setlength\parindent{0pt}


\begin{document}

\maketitle

\tableofcontents

\newpage

\section{第一题解答}

\begin{pinked}
    矩阵的零空间, 解空间, 秩, 以及相抵等性质与域的选取无关.
\end{pinked}

\begin{remark}
    一个常见的错误是出现 $\frac 12$. 在一般的域中, 能否定义 $\frac 12$? 
\end{remark}

\begin{qqq}
    交换 $A$ 的 $[i,j]$ 两行, 等价于左乘一个矩阵 $S_{i,j}$. 写出该矩阵. 
    \begin{aaa}
        假定 $i<j$. 以下斜逗号 $\ddots$ 处为 $1$, 空白处为 $0$. 
\begin{equation}
    \begin{matrix}
        \\[-6pt]
       \text{行 }\, i\rightarrow \\[-6pt]
        \\[-6pt]
       \text{行 }\, j\rightarrow \\[-6pt]
        \\[-6pt]
       \end{matrix} \ \begin{pmatrix}
       1 &  &  &  &  &  & \\[-6pt]
        & \ddots  &  &  &  &  & \\[-6pt]
        &  & 0 &  & 1 &  & \\[-6pt]
        &  &  & \ddots  &  &  & \\[-6pt]
        &  & 1 &  & 0 &  & \\[-6pt]
        &  &  &  &  & \ddots  & \\[-6pt]
        &  &  &  &  &  & 1
       \end{pmatrix}.
\end{equation}
简化地, $S_{i,j}=I-E_{i,i}-E_{j,j}+E_{i,j}+E_{j,i}$. 
    \end{aaa}
\end{qqq}

\begin{qqq}
    将 $A$ 第 $k$ 行的各项同时乘上一个非零常数 $\lambda$, 等价于左乘一个矩阵 $D_k^{\lambda}$. 写出该矩阵.
    \begin{aaa}
        以下斜逗号 $\ddots$ 处为 $1$, 空白处为 $0$. 
\begin{equation}
    \begin{matrix}
        \\[-6pt]
       \text{行 } k\rightarrow \\[-6pt]
        \\[-6pt]
       \end{matrix} \ \begin{pmatrix}
       1 &  &  &  & \\[-6pt]
        & \ddots  &  &  & \\[-6pt]
        &  & \lambda  &  & \\[-6pt]
        &  &  & \ddots  & \\[-6pt]
        &  &  &  & 1
       \end{pmatrix}.
\end{equation}
简化地, $D_k^\lambda=I+(\lambda -1)E_{k,k}$. 
    \end{aaa}
\end{qqq}

\begin{qqq}
    向 $A$ 的第 $j$ 行加上其第 $i$ 行的 $\lambda$ 倍 (这一过程仅改变第 $j$ 行, 其他行不变), 等价于左乘一个矩阵 $T_{i,j}^\lambda$. 写出该矩阵.
    \begin{aaa}
        不妨设 $i<j$. 以下斜逗号 $\ddots$ 处为 $1$, 空白处为 $0$. 
\begin{equation}
    \begin{matrix}
        \\[-6pt]
       \text{行 } i\rightarrow \\[-6pt]
        \\[-6pt]
       \text{行 } j\rightarrow \\[-6pt]
        \\[-6pt]
       \end{matrix} \ \begin{pmatrix}
       1 &  &  &  &  &  & \\[-6pt]
        & \ddots  &  &  &  &  & \\[-6pt]
        &  & 1 &  &  &  & \\[-6pt]
        &  &  & \ddots  &  &  & \\[-6pt]
        &  & \lambda &  & 1 &  & \\[-6pt]
        &  &  &  &  & \ddots  & \\[-6pt]
        &  &  &  &  &  & 1
       \end{pmatrix}.
\end{equation}
简化地, $T_{i,j}^\lambda =I+\lambda E_{j,i}$. 
    \end{aaa}
\end{qqq}

\begin{qqq}
    求逆变换 (逆矩阵) $S_{i,j}^{-1}$, $(D_k^\lambda)^{-1}$, 以及 $(T_{i,j}^\lambda)^{-1}$. 
    \begin{aaa}
        依次解答如下. 
        \begin{enumerate}
            \item 注意到 $(S_{i,j})^2=I$, 故 $S_{i,j}^{-1}=S_{i,j}$.
            \item 注意到 $(D_k^\lambda)^{-1}=D_k^{1/\lambda}$.
            \item 注意到 $(T_{i,j}^\lambda)^{-1}=T_{i,j}^{-\lambda}$. 
        \end{enumerate}
    \end{aaa}
\end{qqq}

\begin{qqq}
    使用自然语言描述这三类逆变换. 
    \begin{aaa}
        依次回答如下. 
        \begin{enumerate}
            \item $S_{i,j}^{-1}$ 的定义: 换回 $[i,j]$ 两行, 也就是交换矩阵的 $[i,j]$ 两行.
            \item $(D_k^\lambda)^{-1}$ 的定义: 将矩阵的第 $k$ 行由数乘后的结果复原, 也就是在第 $k$ 行各项同时乘上非零常数 $1/\lambda$. 
            \item $(T_{i,j}^\lambda)^{-1}$ 的定义: 在矩阵的第 $j$ 行中剔除第 $i$ 行的 $\lambda$ 倍, 也就是向 $A$ 的第 $j$ 行加上其第 $i$ 行的 $-\lambda$ 倍 (这一过程仅改变第 $j$ 行, 其他行不变).
        \end{enumerate}
    \end{aaa}
\end{qqq}

\begin{qqq}
    求 $S_{i,j}S_{k,l}=S_{k,l}S_{i,j}$ 的充要条件. 
    \begin{aaa}
        由构造, $S_{i,j}$ 与 $S_{j,i}$ 无区别, $S_{i,i}=I$. 为找到以上两个矩阵乘积不交换的充要条件, 只需讨论 $i\neq j$ 且 $k\neq l$ 的三种情形. 
        \begin{enumerate}
            \item $(i,j,k,l)$ 两两不等. 
            \item $(i,j,k)$ 两两不等, $i=l$. 
            \item $i=k$ 且 $j=l$. 
        \end{enumerate}
只有第二种情形是非交换的. 记 $|S|$ 为有限集的大小, 则第二种情况等价于
\begin{equation}
    |\{i,j\}|=2\,\text{ 且 } \,|\{k,l\}|=2\,\text{ 且 }\, |\{i,j,k,l\}|=3.
\end{equation}
换言之, 以上乘积可交换的充要条件是
\begin{itemize}
    \item 两个矩阵的``作用范围''全不交, 或一者包含另一者. 
\end{itemize}
    \end{aaa}
\end{qqq}

\begin{qqq}
    求 $T_{i,j}^\lambda T_{k,l}^\mu=T_{k,l}^\mu T_{i,j}^\lambda$ 的充要条件. 
    \begin{aaa}
        依照 $T_{i,j}^\lambda =I+\lambda E_{j,i}$, 得
    \begin{align}
        T_{i,j}^\lambda T_{k,l}^\mu-T_{k,l}^\mu T_{i,j}^\lambda&=(I+\lambda E_{j,i})(I+\mu E_{l,k})-(I+\mu E_{l,k})(I+\lambda E_{j,i})\\[6pt]
        &=\lambda\mu (E_{j,i}E_{l,k}-E_{l,k}E_{j,i}).
        \end{align}
上式为 $0$ 的充要条件是以下任意一者成立:
\begin{enumerate}
    \item $\lambda \mu=0$; 
    \item $i\neq j$ 且 $k\neq j$ (两处 $0$ 相减); 
    \item $i=j=k=l$ (两处 $1$ 相减). 
\end{enumerate}
    \end{aaa}
\end{qqq}

\begin{qqq}
    以上给出了三类矩阵. 能否通过某两类矩阵得到第三类? 请讨论这三种情况 (构造或给出反例). 
    \begin{itemize}
        \item 另注: ``给出反例''的本质还是证明. 原问题表述欠妥. 
    \end{itemize}
    \begin{aaa}
        $S$ 类能通过 $T$ 类与 $D$ 类得到, 其机理类似赋值 $(a,b):=(b,a)$. 记 
\begin{equation}
    A:=D_i^{-1}\cdot T_{i,j}^1\cdot T_{j,i}^{-1}\cdot T_{i,j}^1,
\end{equation}
   此时 $A$ 的左乘表现为以下的映射的合成: 
\begin{equation}
    (x_i,y_j)\xrightarrow{T_{i,j}^1} (x_i,x_i+y_j)\xrightarrow{T_{j,i}^{-1}}(-y_j,x_i+y_j)\xrightarrow{T_{i,j}^1}(-y_j,x_i)\xrightarrow{D_{i}^{-1}}(y_j,x_i).
\end{equation}
    \end{aaa}

    $T$ 类不能通过 $S$ 类与 $D$ 类得到. 因为 $S$ 类矩阵与 $D$ 类矩阵的任意乘积都至少有 $n^2-n$ 个零. 

    若假定 $T_{i,i}^\lambda$ 是合法的, 则 $D$ 类矩阵能通过 $S$ 类与 $T$ 类得到. 因为 $D_{i}^\lambda=T_{i,i}^{\lambda-1}$. 

    若假定 $T_{i,i}^\lambda$ 是非法的 (假定 $n\geq 2$), 则有且仅有行列式为 $\pm 1$ 的 $D$ 类矩阵能被其余两类矩阵表示. 对行列式为 $-1$ 的 $D$ 类矩阵进行如下 $S$-类与 $T$-类的初等行变换即可: 
    \begin{equation}
        \begin{pmatrix}
            1 & 0\\
            0 & -1
            \end{pmatrix} \rightsquigarrow \begin{pmatrix}
            0 & -1\\
            1 & 0
            \end{pmatrix} \rightsquigarrow \begin{pmatrix}
            0 & -1\\
            1 & 1
            \end{pmatrix} \rightsquigarrow \begin{pmatrix}
            1 & 0\\
            1 & 1
            \end{pmatrix} \rightsquigarrow \begin{pmatrix}
            1 & 0\\
            0 & 1
            \end{pmatrix}.
    \end{equation}
    若 $D$ 类矩阵的行列式不为 $\pm 1$, 则其无法通过 $S$ 类与 $T$ 类矩阵的乘积得到. 因为 $\det (AB)=\det (A)\det (B)$. 
    \begin{itemize}
        \item 对于二元域或三元域上的矩阵, $D$ 类矩阵可以通过其他两类矩阵得到. 
    \end{itemize}
\end{qqq}

\begin{qqq}
    假定 $A$ 是方阵. 将以上 $S_{i,j}$, $T_{i,j}$ 与 $D_k$ 乘在 $A$ 的右侧, 效果如何? 
    \begin{aaa}
        解答如下. 
        \begin{enumerate}
            \item $S_{i,j}$ 右乘: 交换 $A$ 的 $i$ 列与 $j$ 列; 
            \item $T_{i,j}^\lambda$ 右乘: 将 $i$ 列的 $\lambda$ 倍加至 $j$ 列; 
            \item $D_k^{\lambda}$ 右乘: 将第 $k$ 列乘以 $\lambda$. 
        \end{enumerate}
    \end{aaa}
\end{qqq}

\section{第二题解答}

\begin{qqq}
    给定矩阵 $A$, 其最简行阶梯形 $R$ 为何唯一? 
    \begin{aaa}
        见下一问. 
    \end{aaa}
\end{qqq}

\begin{qqq}
    (接上一小问) 尝试给出一个不用计算的证明 (无字证明). 
    \begin{aaa}
        从 $A$ 到 $\mathrm{rref}(A)$ 可通过``改写矩阵各列''实现, 无需进行复杂的初等行变换.

        以下算法从左往右地读取 $A$ 的各列, 并逐列给出 $\mathrm{rref}(A)$. 记 $A$ 共有 $n$ 列, 且第 $k$ 列为向量 $v_k$. 
\begin{quote}
    输入: 矩阵 $A$ 各列 $\{v_1,v_2,\ldots, v_n\}$; 

    输出: 最简行阶梯形 $\mathrm{rref}(A)$ 各列 $\{c_1,c_2,\ldots, c_n\}$. 

    对于 $k=1,2,\ldots, (A \text{ 的列数})$.

若 $v_k\in \mathrm{Span}(\{v_k\}_{t<k})$, 则 $c_k$ 的坐标由 $\{c_t\}_{t<k}$ 唯一确定. 

若不然, 则取 $c_k$ 是为 $\{0,1\}$-向量 $e_{t+1}$, 其中 $t=\dim \mathrm{Span}(\{v_k\}_{t<k})$. 
\end{quote}
    \end{aaa}
    注: 请回忆 $\mathrm{Span}(\varnothing)=0$, 从而以上算法是精简且统一的. 空集起到了 \verb|void| 的效果. 
\end{qqq}

\begin{qqq}
    转置矩阵 $R^T$ 的最简行阶梯形是什么? 
    \begin{aaa}
        这等价于说, 先在 $R$ 中消去所有 $1$ (阶梯拐角) 右侧的项, 再将所有零向量移至非零向量的右侧. 因此, $R^T$ 的最简阶梯型是
\begin{equation}
    \begin{pmatrix}I_r&O\\O&O\end{pmatrix}.
\end{equation}
    \end{aaa}
\end{qqq}

\begin{qqq}
    证明相抵标准型的存在性: 对任意矩阵 $A$, 存在可逆矩阵 $P$ 和 $Q$ 使得
\begin{equation}
    A=P\begin{pmatrix}I_r&O\\O&O\end{pmatrix}Q.
\end{equation}
以上, $I_r$ 是 $r$ 阶单位矩阵, $O$ 表示数字 $0$ 出现的位置. 

注意: 中间的 $(0,1)$-矩阵兼并了以下三类退化矩阵
\begin{equation}
    \begin{pmatrix}I_r&O\end{pmatrix},\quad \begin{pmatrix}I_r\\O\end{pmatrix},\quad \begin{pmatrix}I_r\end{pmatrix}.
\end{equation}
\begin{aaa}
    上一问已给出答案. 
\end{aaa}
\end{qqq}

\begin{qqq}
    证明以上的 $r$ 由 $A$ 唯一决定. 作为推论, 矩阵的行秩等于列秩. 往后统一称作秩. 
    \begin{pinked}
        此类分解不必来自矩阵的初等行列变换, 从而不能使用消元法的唯一性证明. 
        
        在证明相抵标准型的唯一性之前, 使用各类``秩不等式''或有循环论证之嫌. 
        
        最好的方法是反证法. 
    \end{pinked}
    \begin{aaa}
        若存在可逆的 $P_1,P_2,Q_1,Q_2$ 使得
\begin{equation}
    A=P_1\begin{pmatrix}I_r&O\\O&O\end{pmatrix}Q_1=P_2\begin{pmatrix}I_s&O\\O&O\end{pmatrix}Q_2.
\end{equation}
若 $r<s$, 改写等式得
\begin{equation}
    P_2^{-1}P_1\begin{pmatrix}I_r&O\\O&O\end{pmatrix}Q_1Q_2^{-1}=\begin{pmatrix}I_s&O\\O&O\end{pmatrix}.
\end{equation}
考虑 $\begin{pmatrix}I_r&O\\O&O\end{pmatrix}=\begin{pmatrix}I_r\\O\end{pmatrix}\cdot \begin{pmatrix}I_r&O\end{pmatrix}$, 则上式改写作
\begin{equation}
    M\cdot N=\begin{pmatrix}I_s&O\\O&O\end{pmatrix}\quad (M\in \mathbb F^{m\times r},\, N\in \mathbb F^{r\times n}).
\end{equation}
将矩阵左乘视作 $\mathbb F^n$ 至 $\mathbb F^m$ 的线性映射, 则
\begin{itemize}
    \item 左乘 $M\cdot N$ 必然将 $s$ 个线性无关的行向量映作线性相关组 (因为 $N$ 总共仅有 $r$ 行); 
    \item 考虑 $\{e_1,\ldots, e_s\}$, 则左乘 $\begin{pmatrix}I_s&O\\O&O\end{pmatrix}$ 保持这一线性无关组.
\end{itemize}
综上, 原等式矛盾. 因此 $r\geq s$. 类似的论证给出 $r\leq s$, 从而 $r=s$. 
    \end{aaa}
\end{qqq}

\section{``挑战题''解答}

\begin{qqq}
    若整数矩阵 $A=\begin{pmatrix}a&b\\c&d\end{pmatrix}$ 满足 $ad-bc=1$, 则 $A$ 是以下几类矩阵的有限乘积
\begin{equation}
    S = \begin{pmatrix}0 & 1 \\ -1 & 0\end{pmatrix}, \qquad  T = \begin{pmatrix}1 & 1 \\ 0 & 1\end{pmatrix},\qquad T^{-1} = \begin{pmatrix}1 & -1 \\ 0 & 1\end{pmatrix}.
\end{equation}
\begin{aaa}
    依题设, 直接得到 $T^k$ ($k\in \mathbb Z$). 因此可以将 $A$ 第二行的任意倍数加至第一行. 计算得
\begin{equation}
    STS= \begin{pmatrix}1 & 0 \\ 1 & 1\end{pmatrix},\qquad ST^{-1}S= \begin{pmatrix}1 & 0 \\ -1 & 1\end{pmatrix}.
\end{equation}
因此可以将 $A$ 第一行的任意倍数加至第二行. 若考虑右乘, 则可以将 $A$ 某列的任意倍数加至另一列. 同时, $S$ 起到了行置换与列置换的作用. 
\begin{itemize}
    \item 若 $A$ 不存在 $0$, 则不妨设 $A$ 的第一列非零. 依照行列式, 该列的两数互质. 通过上述变换, 可将 $A$ 中某项消作 $0$. 
    \item 若 $A$ 中存在一个 $0$, 则可以通过上述变换将 $A$ 变作上三角矩阵. 依照行列式, 两个对角元只能是 $\pm 1$. 依照初等变换, 可以将 $A$ 变作对角矩阵. 
    \item 若 $A$ 中存在两个零, 则 $S$ 或 $SA$ 是对角矩阵. 最后依照 $S^2=-I$, 明所欲证. 
\end{itemize}
\end{aaa}
\end{qqq}






\end{document}