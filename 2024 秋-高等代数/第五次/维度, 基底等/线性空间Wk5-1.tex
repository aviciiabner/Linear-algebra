\documentclass[11pt]{ctexart}

\usepackage[margin=2cm,a4paper]{geometry}
\usepackage{amsthm, amsfonts, amsmath, amssymb, mathrsfs, newclude, tikz-cd, tikz, ctex, mathtools, stmaryrd, datetime}


%\setmainfont{Caladea}

%% 也可以选用其它字库:
% \setCJKmainfont[%
%   ItalicFont=AR PL KaitiM GB,
%   BoldFont=Noto Sans CJK SC,
% ]{Noto Serif CJK SC}
% \setCJKsansfont{Noto Sans CJK SC}
% \renewcommand{\kaishu}{\CJKfontspec{AR PL KaitiM GB}}



\usepackage[colorlinks = true,
linkcolor = blue,
urlcolor  = blue,
citecolor = blue,
anchorcolor = blue]{hyperref}

% Include the x-color package for color support
\usepackage{xcolor}

% Define a new environment for red comments
\usepackage{verbatim} % Required for the comment environment
\usepackage{environ}

\usepackage{mdframed} % Include mdframed for creating framed environments

\definecolor{pinked}{RGB}{255,231,229} % Define a base color 
% Define a new environment with a background color
\newmdenv[
  backgroundcolor=pinked, % Set the desired background color
  linecolor=white, % Optional: Set the border line color
  linewidth=1pt, % Optional: Set the border line width
  roundcorner=5pt, % Optional: Set rounded corners
  nobreak=true % Optional: Prevent page breaks within the environment
]{pinked}

\theoremstyle{definition}
\newtheorem{qqq}{问题}[section]

\newcommand{\ExternalLink}{%
    \tikz[x=1.2ex, y=1.2ex, baseline=-0.05ex]{% 
        \begin{scope}[x=1ex, y=1ex]
            \clip (-0.1,-0.1) 
                --++ (-0, 1.2) 
                --++ (0.6, 0) 
                --++ (0, -0.6) 
                --++ (0.6, 0) 
                --++ (0, -1);
            \path[draw, 
                line width = 0.5, 
                rounded corners=0.5] 
                (0,0) rectangle (1,1);
        \end{scope}
        \path[draw, line width = 0.5] (0.5, 0.5) 
            -- (1, 1);
        \path[draw, line width = 0.5] (0.6, 1) 
            -- (1, 1) -- (1, 0.6);
        }
    }

\NewEnviron{aaa}{~\\
    \noindent {\textcolor{teal}{\textbf{解答}} \BODY }
}

\NewEnviron{llll}{
    \noindent {~\\$\ExternalLink$ 外部链接 $\,\,\,$ \color{blue}\url{\BODY} }
}

\renewcommand{\proofname}{证明}
\renewcommand\qedsymbol{${\boxed{\substack{\textit{完证}\\\textit{毕明}}}}$}
\let\oldproof\proof
\renewcommand{\proof}{\color{blue}\oldproof}

% Change equation numbering to include the section number
\usepackage{cleveref}
\renewcommand{\theequation}{\thesection.\thesubsection.\arabic{equation}}
\numberwithin{equation}{section}


% 创建适合 python 语言的公式环境. 
\usepackage{pythonhighlight}






\theoremstyle{definition}
\newtheorem*{definition}{定义}
\newtheorem*{proposition}{命题}
\newtheorem*{theorem}{定理}
\newtheorem*{notation}{记号}
\newtheorem*{example}{例子}
\newtheorem{exercise}{\textcolor{blue}{习题}}
\theoremstyle{remark}
\newtheorem*{remark}{备注}
\newtheorem*{lemma}{引理}
\newtheorem*{corollary}{推论}

\title{零散的习题: 线性空间}
\author{(2024-2025-1)-MATH1405H-02}

\setcounter{page}{0}

\setlength\parindent{0pt}

\begin{document}

\maketitle

\vspace{5cm}

\begin{center}\huge
    请完成 \textcolor{blue}{\textbf{习题}} $2^k$ ($k\in \mathbb N_+$). 
\end{center}

\newpage

\section{四大基本空间}

我们目前仅学习了单一矩阵的四大基本空间. 以下是一些推荐读物与参考资料: 
\begin{enumerate}
    \item \href{https://math.mit.edu/~gs/linearalgebra/ila6/ila6_3_5.pdf}{\S 3.5, Strang 的线性代数 (第六版)},
    \item \href{http://mlwiki.org/index.php/Four_Fundamental_Subspaces}{一张清单} (稍微涉及了奇异值分解), 
    \item \href{https://ajitmathsoft.wordpress.com/wp-content/uploads/2019/07/linalg_sage.pdf}{此文第五章} 给出 $\mathsf{Sage}$ 的计算示例 (可使用\href{https://sagecell.sagemath.org/}{临时在线窗格}), 
    \item \href{www.cfm.brown.edu/people/dobrush/cs52/Mathematica/Part3/four.html}{此网页}给出 mathematica 计算示例 (如果你习惯 mathematica). 
\end{enumerate} 

假若学习了奇异值分解, 则可以深入研究 $P(A)$ 与 $P(B)$ 的运算 ($P\in \{C(-),C(-^T),N(-),N(-^T)\}$). 

\newpage 

\section{示例: 通过 Sage 计算 LU 分解}

\begin{exercise}[广义 $LU$-分解]
    假定你证明了 Gauss 消元法存在性. \textcolor{teal}{尽可能简单地}证明: 任意矩阵 $A\in \mathbb F^{m\times n}$ 可以分解作 $A=LS\widetilde IDU$ 的五元乘积形式, 或是 $A=LD\widetilde ISU$ 的五元乘积形式. 此处 
    \begin{enumerate}
        \item $L\in \mathbb F^{m\times m}$ 是主对角为 $1$ 的下三角方阵, 例如 $\begin{pmatrix}
            1&0&0\\a&1&0\\b&c&1
        \end{pmatrix}$; 
        \item $U\in \mathbb F^{n\times n}$ 是主对角为 $1$ 的上三角方阵, 例如 $\begin{pmatrix}
            1&x\\0&1
        \end{pmatrix}$; 
        \item $S$ 是置换方阵, 即先前作业提及的``$S$-类初等变换方阵''; 
        \item $D$ 是对角方阵, 即, $D$ 中非对角元都是 $0$; 
        \item $\widetilde I$ 是相抵标准型的中项. 
    \end{enumerate}
    证明任意一种情形即可, 因为这两种分解仅相差一个转置. 
\end{exercise}

\begin{pinked}
    自主思考: 以上分解在``何种意义下''是唯一的? 
\end{pinked}

\begin{remark}
    ``概率''地, 假定 $A$ 实数域或复数域上的``随机''方阵, 则 $S=I$ 依概率 $1$ 发生. 
\end{remark}

\begin{pinked}
    假定你已经知道了 $PA=LU$ 分解的一般方法, 但疏于计算, 不考虑以下. 
\end{pinked}

\begin{example}
    如果想多做一些题目, 可以使用计算软件进行编题与解题. 
    \begin{enumerate}
        \item[S0] 使用 \href{https://sagecell.sagemath.org/}{$\mathsf{sage}$ 在线窗格} (或者其他方式) 创建 $\mathbb Q$-上的矩阵
        \begin{python}
            A = matrix(QQ, [
                [ 1,  1,  4,  5,  1,  4,  0,  0,  1],
                [-1,  9, -1, -9,  8, -1,  0, -7, -7],
                [ 1,  2, -3, -4,  5,  6, -7, -8,  9],
                [ 3,  1,  4,  1,  5,  9,  2,  6,  5],
                [ 2,  7,  1,  8,  2,  8,  1,  8,  3]
            ]); # Create $A\in \mathbb Q^{m\times n}$. 
                \end{python}
                若想查看矩阵 $A$, 另起新行并键入 \verb|A|, 并点击 \verb|Evaluate| 按钮即可 (快捷键 \verb|Ctrl+Shift+Enter|).
                \item 为查看 $A$ 的最简行阶梯形, 键入 \verb|A.rref()| 并运行即可. 
                \item 广义 $LU$-分解的形式是 $A=PLU$, 键入 \verb|P, L, U = A.LU();| 即可对 $A$ 的 $LU$-分解进行赋值. 
                \begin{itemize}
                    \item 依照 $P^2=I$, 以上即是 $PA=LU$ 分解. 
                \end{itemize}
                \item 若想知道主元的位置, 可键入 \verb|A.pivots()|. 
                \item 自行探索更多. 
    \end{enumerate}
\end{example}

\newpage

\section{线性空间, 基的证明题}

\begin{pinked}
    如果想操练计算题, 可参考``国庆作业''. 
\end{pinked}

\begin{exercise}
    假定 $V$ 任意域 $\mathbb F$ 上的 $2024$ 维线性空间. 试构造子集 $S\subset V$ (向量组), 其同时满足
    \begin{enumerate}
        \item 集合 $S$ 的大小是 $2025$,
        \item $S$ 中任意 $2024$ 个向量线性无关. 
    \end{enumerate}
\end{exercise}

\begin{exercise}[Challenging]
    若 $\mathbb F$ 是数域, 则上题的条件 1 可以放宽至无限集. (What if $\mathbb F$ is finite?)
\end{exercise}


\begin{exercise}[必做的证明题]
    给定数域上的线性空间 $V$. 任意给定 $V$ 的有限个真子空间 $\{U_i\}_{i=1}^m$, 总有 
    \begin{equation}
        \left(\bigcup_{i=1}^m U_i\right)\neq V. 
    \end{equation}
    (若 $\mathbb F$ 非数域, 试给出 $m=3$ 的反例?\footnote{Might there be a one-line counter-example for those who are familiar with $\mathbb F_2$-field?})
\end{exercise}

\begin{exercise}[Challenging]
    在上一习题中置 $m=2$, 则域 $\mathbb F$ 无限制; 
\end{exercise}

\begin{exercise}[如果先前做错了, 请重试]
    若 $U$, $V$ 与 $W$ 是三个子空间, 证明以下等式的一侧, 并证伪另一侧
    \begin{enumerate}
        \item $(U+V)\cap W = (U\cap W) + (V\cap W)$; 
        \item $(U\cap V)+W = (U+W) \cap (V+W)$. 
    \end{enumerate} 
\end{exercise}

\begin{exercise}[如果先前做错了, 请重试]
    若 $U\subset V$ 与 $W$ 是三个子空间, 证明 $(U+W)\cap V = U+(W \cap V)$. 
\end{exercise}

\begin{exercise}
    根据上述习题, 证明以下两个等式. 选定 $U$, $V$ 与 $W$ 为同一线性空间的三个子空间, 试证明: 
    \begin{enumerate}
        \item $((V \cap W) + U) \cap V = {\color{white}{ (V \cap W) + (U\cap V) }} = V \cap (W + (U\cap V))$, 
        \item $((V + W) \cap U) + V = {\color{white}{ (V + W) \cap (U + V) }} = V + (W \cap (U + V))$. 
    \end{enumerate}
    关键步骤是中间白色处. 
\end{exercise}

\newpage 

\section{\large (span : 子集 \texorpdfstring{$\to$}{PDFstring} 子空间)  (dim : 子空间 \texorpdfstring{$\to$}{PDFstring} \texorpdfstring{$\mathbb N$}{PDFstring} ) 与 (rank = dim \texorpdfstring{$\circ$}{PDFstring} span)}

\begin{notation}
    谈及 $\dim$ 与 $\mathrm{rank}$, 默认``参与关键运算''的线性空间是有限维的. 
\end{notation}

\begin{pinked}
    以下定义, 定理, 以及习题等的表述是更偏类型化的: 这兼顾了严谨性与简易性. 
\end{pinked}

\begin{definition}[``\texorpdfstring{$\mathrm{rank}=\mathrm{dim}\circ \mathrm{span}$}{}'']
    我们形式化地澄清三个记号. 以下谈论的线性空间都附带了域. 
    \begin{enumerate}
        \item[$\mathrm{span}$] $\texttt{输入\_1}$ 是线性空间 $V$, $\texttt{输入\_2}$ 是 $V$ 的子集 $S$;  
        
        $\texttt{输出}$ 是 $V$ 中一切包含 $S$ 的线性子空间之交. 
        \begin{exercise}
            需要证明, $\texttt{输出}$ 也是线性空间, 并恰是包含 $S$ 的 $V$-线性子空间中的极小者.
        \end{exercise}

        \item[$\mathrm{dim}$] $\texttt{输入}$ 是有限维线性空间 $V$; 

        $\texttt{输出}$ 是自然数 $n$, 即 $V$ 中任一极大线性无关组的大小. 
        \begin{exercise}
            需要证明, 任选定 $V$ 中任意两组极大线性无关组, 其作为集合大小相同.
        \end{exercise}

        \item[$\mathrm{rank}$] 当且仅当 $\mathrm{span}$ 输出有限维线性空间, 方可定义 $\mathrm{rank}=\mathrm{dim}\circ \mathrm{span}$. 
    \end{enumerate}
\end{definition}

\begin{pinked}
    固定 $\texttt{输入\_1}$, 以下研究 $\texttt{输入\_2}$ 的变化对以上的影响. 最简单的子集关系是包含. 
\end{pinked}

\begin{exercise}[保序]
    $\dim$ 与 $\mathrm{rank}$ 保持特定的序关系. 给定全空间 $V$ 与子集 $S_1, S_2\subset V$, 考虑下图: 
    \begin{equation}
        % https://q.uiver.app/#q=WzAsOSxbMCwxLCJbU18xLFZdIl0sWzAsMiwiW1NfMixWXSJdLFsyLDEsIlxcbWF0aHJte3NwYW59X1YoU18xKSJdLFsyLDIsIlxcbWF0aHJte3NwYW59X1YoU18yKSJdLFs0LDEsIlxcbWF0aHJte3Jhbmt9KFNfMSkiXSxbNCwyLCJcXG1hdGhybXtyYW5rfShTXzIpIl0sWzAsMCwiY18xIl0sWzIsMCwiY18yIl0sWzQsMCwiY18zIl0sWzAsMl0sWzIsNF0sWzEsM10sWzMsNV0sWzAsMSwiXFxzdXBzZXQgIiwzLHsic3R5bGUiOnsiYm9keSI6eyJuYW1lIjoibm9uZSJ9LCJoZWFkIjp7Im5hbWUiOiJub25lIn19fV0sWzIsMywiXFxzdXBzZXQgIiwzLHsic3R5bGUiOnsiYm9keSI6eyJuYW1lIjoibm9uZSJ9LCJoZWFkIjp7Im5hbWUiOiJub25lIn19fV0sWzQsNSwiXFxzdXBzZXQgIiwzLHsic3R5bGUiOnsiYm9keSI6eyJuYW1lIjoibm9uZSJ9LCJoZWFkIjp7Im5hbWUiOiJub25lIn19fV1d
\begin{tikzcd}[ampersand replacement=\&]
	{c_1} \&\& {c_2} \&\& {c_3} \\
	{[S_1,V]} \&\& {\mathrm{span}_V(S_1)} \&\& {\mathrm{rank}(S_1)} \\
	{[S_2,V]} \&\& {\mathrm{span}_V(S_2)} \&\& {\mathrm{rank}(S_2)}
	\arrow[from=2-1, to=2-3]
	\arrow["{\supset }"{marking, allow upside down}, draw=none, from=2-1, to=3-1]
	\arrow[from=2-3, to=2-5]
	\arrow["{\supset }"{marking, allow upside down}, draw=none, from=2-3, to=3-3]
	\arrow["{\supset }"{marking, allow upside down}, draw=none, from=2-5, to=3-5]
	\arrow[from=3-1, to=3-3]
	\arrow[from=3-3, to=3-5]
\end{tikzcd}.
    \end{equation}
    基于对称性, $\subset$ 可以表示子集的包含, 线性子空间的包含, 自然数的小于等于号. ``保序''是说, 
    \begin{itemize}
        \item 若 $c_i$ 处的 $\subset$ 成立, 则 $c_{i+1}$ 处的 $\subset$ 亦成立. 
    \end{itemize}
\end{exercise}

\begin{exercise}
    证明以下问题. 
    \begin{enumerate}
        \item 若 $c_1$ 取 $\subset$, 则 $c_3$ 取等当且仅当 $c_2$ 取等. 
        \item 若 $c_i$ 取等, 则 $c_{i+1}$ 亦然. 
        \item 若 $c_{i+1}$ 取等, 则 $c_i$ 不必取 $\subset$. 
    \end{enumerate}
     
\end{exercise}

\begin{exercise}
    思考平凡情况: $S=\emptyset$ (理解作 $\mathrm{void}$) 或 $S=V$ (作为集合, 理解作 $S=V.\text{Set}$). 
\end{exercise}

\begin{pinked}
    下一步是建立二元运算. 暂时将 $\subset$ 区分地记作 $\subseteq$ (集合), $\subset$ (子空间), $\leq$ (自然数). 
\end{pinked}

\begin{example}
    自然的想法是下述表格 (子空间的交记作 $\wedge$, 以区别于集合的交 $\cap$): 
    \begin{equation}
            \begin{matrix}
                \text{名称} & \text{序} & \text{极大} & \text{极小} & \text{所谓 And} & \text{所谓 Or}\\\hline
                \text{子集} & \subseteq  & V_{\text{Set}} & \emptyset  & \cap  & \cup \\
                \text{子空间} & \subset  & V & \mathbf 0 & \wedge  & +\\
                \text{秩} & \leq  & \text{NA} & 0 & \min & \max
                \end{matrix}.
    \end{equation}    
\end{example}

\begin{exercise}
    选用 $S_1\subseteq S_2$, 则有
    \begin{itemize}
        \item $\mathrm{span}(\emptyset)=0$, $\mathrm{span}(S_1\cap S_2) = \mathrm{span}(S_1)\cap \mathrm{span}(S_2)$; 
        \item $\mathrm{span}(V) = V$, $\mathrm{span}(S_1\cup S_2) = \mathrm{span}(S_1)+ \mathrm{span}(S_2)$. 
    \end{itemize}
    选用 $U_1\subset U_2$, 则有 (容易补全 $\mathrm{rank}$-方向...)
\end{exercise}

\begin{pinked}
    若所谈论的对象构成全序 (等价地, 只看一条链), 则以上三类偏序关系是逐次的商集. 
\end{pinked}

\begin{exercise}
    给定子空间 $U_1$ 与 $U_2$. 
    \begin{enumerate}
        \item 证明 $\mathrm{rank}(U_1\wedge U_2)\leq \min (\mathrm{rank}(U_1),\mathrm{rank}(U_2))$; 
        \item 证明 $\max (\mathrm{rank}(U_1),\mathrm{rank}(U_2))\leq \mathrm{rank}(U_1 + U_2)$. 
    \end{enumerate}
    提示: 第一处仅使用逻辑``或'', 第二处仅使用逻辑``与''; 无关具像之选取. 
\end{exercise}

\begin{remark}
    从高维的``序''降至低维的``序'', 自然省略了诸多信息. 
\end{remark}

\begin{exercise}
    给定子集 $S_1$ 与 $S_2$. 
    \begin{enumerate}
        \item 证明 $\mathrm{span}(S_1\cap S_2)\subset \mathrm{span}(S_1)\wedge \mathrm{span}(S_2)$. 
        \item 证明 $\mathrm{span}(S_1) + \mathrm{span}(S_2)=\mathrm{span}(S_1\cup S_2)$. 
    \end{enumerate}
\end{exercise}

\begin{remark}
    问题出在何出? 此处的问题是下式``为何取等'', 而非上式为何不等. 
\end{remark}

\begin{proposition}
    对同一集合的三个子集 $G$, $H$ 与 $U$, 总有
    \begin{enumerate}
        \item $G \cup (H \cap U) = (G \cup H) \cap (G \cup U)$, 与
        \item $(G \cap H ) \cup ( G \cap U ) = G \cap (H \cup U)$. 
    \end{enumerate}
    \begin{proof}
        由于 $x\in X$ 是命题, $(\star \in X)\,\text{and}\,(\star \in Y$) 当且仅当 $\star \in (X\cap Y)$, 以及 $(\star \in X)\,\text{or}\,(\star \in Y)$ 当且仅当 $\star\in X\cup Y$. 鉴于此, 我们可以在不使用 \verb|mathlib| 的情况下用 $\mathsf{L\exists \forall N}$ 进行形式化的证明, 见\href{https://live.lean-lang.org/#codez=LTAEBEEsGcBcCdICMCutIHsB2oAO8NcBTeWAT1AHFRAKIlAAlRByIlAFVRAUwlAApq76AlMx582AgHQAoWAAsiGeEQC2EGAmRpMWAPoAmBdoCMAQywATEQzagAXKAAKBXELujGLdl160GQlt7pWFwBeUCQySSIAD2MAY1hQQAvyQG7gGVBAMsJQAHl4cSIAG0gVNO5kikzEnPFILHzQMgAabNyaurJAS/IhUqiMpKqa+FAo8UMm/qxB4d1OptTeqoKi0BkR5uranmSotMyFwuLxXTXWnnHBwAwiUABBc3WEDCGZAQF2yUkQUGjjJVx8oktGOw7I5CF00q4fO4xLZLPw/LDoTZQuF3mAvvE1otiptyn0WhtGscCZ1Nj0KmchiMxvjJodOqjQABJLC4NB2XiAfipuPRAPBEQSaADMbJRAEmElAwxnyTQA5jZuXyBGKJfkBE0AO5yRQM0CgIX1XoAbQoOsAEEQiDmCEYAXVApgsRp1oDNnKC1oA3NrQLKhobhoYnebLbobXbQAapgGXRJg1IPlliPBjOhsK4ubygqARVRlbGwFk0KzYK4cwyvj8/gDrMCnGD2W54QExC4qJDhEDkREPuiEnslit/eM6t0dpj9stDkShxTLjczHcCI9ngzmYW9DDuCsjqFKBbqes6q4lZK9wNQJdEoBpQlANQYDWA7Rh9BF9CYrCPKoaDPzsFXRzsV1AAADAUUCweIFEA0ALzsdgs3FSUGQ1Eh/ivGBbVwX4yCICxYAwZcWTQIx137UBtw5VgTw2Q94KlKCYTfGiBDeD5mVAWIMCUJRsFAaAiCwXimiuAB17gACEAGoAGEBGCYSJOEyTPw+R0VNUx0rkAdKIxMAVKJpNkjTRO0zTFM9NSzOuHkxI5PTuCuHlRIELk7OkpoiFgWJc1AABNbBsOMOwADFQPiWzLNE6yhCzAB+JjlNUrggrA2ArMkzNQESkKAOipc4pUhLguS0S0oy5KstAGKhA5dKCu4VKsxK2y0oq0zHXypLQqK7LQCqhrnKa2KwDUtrMvsjkrhFHk6vKiQGVEihQMgAA3EhoElPAnBIcgmiQDBZB4yAzCIaAeIwfJltAWR/lWpR/nwDAkD+JRoCaOQwP+BaAEcUCIKQGSgOBEFQZMcDuhNyBbfxGECIQuGoFgoYhthfsuhRlFUAGNGBowDF0UNvEBGEQWcGE4asOhSc8RHuR8DMkTCCJuySUA5nJXIsWWa5bj+AUElncRudgPd2ZKQcz05udEGlGRYC6adxfESXpZeWYR17A4jm4UXLls24anucdDGeU4aTFnW5z1hd+2eV5Sxict/nxqsHBrHhwSphGgmbUmEYp2wO1tuIezZsdN2N/dTb5i2Hit2WTe1yOsH1mPmLRO3fgd6gCerUFXZJ4QfbbL3hGp6G/fpgOMUqYO+3lgXa6IHmhZDsOThnW5FZlluJjFvmO5efDCzsUPQnociPVylStaSaU7wfOxGDgnkF+zSVx8GtS5cSGf70fNLKD5PeSwD+3KyBZ2c43PP4Z8Cmi8hmmQnLrsYkr144ywf4ZGMY7Eme2B2nVJqIgKceJXzJojWCVAB6EWOmkUIvAl6OQQUENe5k8AkyXnvVB5kPoYMgZQTyZZ06n0Ji7S+EJr7k0LnnOmKJn6B1HDXPmddmEN1gCAhIFDwG+yzB7aBsBtAJDgSIRBXJ94oJaipXAGDl6j1YNgsyuDXAH14eRQhacKyOzPkTMEYCC4eGbB7MudDU4MLVhzTW+I6htwlpAKWndLHrAuPLPuICAByGA1SgDVP8EE51LqgDMGoQGmhuLwBQBWG8cAUC4AOt44wZBoCeQACpyF1KdfIniajSh4jITxx0AkoGOhgAUbFv5HU8v9dQQMtDrUIJtCgbg2ycARIIYuohXRvCIZozOTsdF534M0rw9Z2kP2MREdi/EEAoHAvASQUdliSEmQDGZuE5mxHKbA7xkBZCSB1AAH2vBsYwpEAB8oABZultNAaAKAfjAz2aAQ5p4kBnNAB3K5GzeLHSQFc7+tz7laAZAFBQudgQ+AAIqqlAL5UARSjoMi+UdDmaodkyE9M8jYWBvhEG0HXN59TExY0MFcwlSYtB6CueIalGKjmDGxTdbQHcCUJnJdgIwpLWVY10FSmlSKtmot2Qco5dQTnBHOZc65ALcAPOFS8t5HyynfLCH8m5dyZVAoWei/lKK0WPMxfrE5rzxXAtBeQhwwgoVNFhfC6AiLNm6t2R8A1C4jVnKWQ64wkgBaSH+eqh5OqkDerYb6tVgLsCSA7ks7AKzZmRqIMQJMUr/VAr+sEzGtTQYNPdmM2G+cb5tl+t0jOVhtEuy4fo6web76l1oRMmN0y41aujVM8JcadVpEFeizFoq3k6uMKq6VDzMWDGNecwNg6U0Ro7ds3ZPbbR9s9VcyVfrw1YEkCOsIi7lW/PeXY6Wk613zMTpbD1yrO16pdQ8N1JrA2zu7SK9B4qLlsM+Q2ttayrmKETQkVdGqI2btwc+xVyzG2fveQmogSa/0PKAA}{\underline{此篇个人草稿}}的 212 行与 214 行. 
\end{proof}
\end{proposition}

\begin{exercise}
    若 $U$, $V$ 与 $W$ 是三个子空间, 证明以下等式的一侧, 并证伪另一侧
    \begin{enumerate}
        \item $(U+V)\wedge W = (U\wedge W) + (V\wedge W)$; 
        \item $(U\wedge V)+W = (U+W) \wedge (V+W)$. 
    \end{enumerate} 
\end{exercise}

\begin{example}
    接上题. 尽管等式不必成立, 但式 $1$ 取等号当且仅当式 $2$ 取等. 
    \begin{proof}
        仍然给一个形式化的证明, 见\href{https://live.lean-lang.org/#codez=JYWwDg9gTgLgBAWQIYwBYBtgCMB0B5KAEwFMocAZFGYAY2LgCgGA3JKYJLdegb0EbgOAC44AFQCeYYgCoAvnADalGNTpw+AXRZsOXegAoAqnABqcAOpC1ASiYBaW3EAXNsMD9foHIDQF5egF9TA9GaBPJ0Ao60AQtzglFWIcRgZ7GNiYwGNrQHDTOJSUuwdABujAd+VhAwAaEzhAODlzOB93QAVtb0j0uEBkojKqryDAErkCwBSiRu8gwBiVGuiHQAA5QBogwAZ1AoagwAX4wHTvQAx5QAlFQHK/QHJNQDi5OHb+tGJoYhA4TgBnaCwARksjBsNNkys4AF44I0EnrDEBuD1ALO1AMBdAUMVAJ3a90q3mecEA3AaAQ7tAM2xgFO5QDQcoBfFUAGRk4T6AT+1AN9ygHhDUFeEwFCGASHNwU8jIBKIi+RgA1HdIow4IcwGB0GI4NwAPpIAB21COYhAIAYTKQLLZcGAPIAZpyudxpTARczWeyuVLpcqxaqOcROVBiNL0FrxWq9UcAK5gOWGpUMHZ7A7HU4AJkutxuDWM93JQjeH3sXz+QJBbia4Oh8ORaMxuPxhMhpKMvrpNyp3oZJp1XN5/MFwtFprglut3CzEq5BqN5fZGptNsVNd1nJLDbtgbhgCg5QAkcoBy+TgfwA9IBUfV+g7cgApYuCAB89ADrygA4VQBspoBo/X6xAAHkhwNwvqhLIAAInBpgswj4BSuX1MnTMPup9TuHovN/uuQ9pgaN8ed863vvN2vL7+p8PwAsCcCAE+6gCn7oAkHKAFRy+KwowgaUpecB0p+Txpg+qYoTeBSfIAmEoFIAJtaAG+mxKAJLegDw+khDi0qhVIYXedK/lSNzoVY6JMlAADuCioJo3F8fIAACdY0BAQpwJy0nfncglwLxCjOlA5wKUponiZJBwyTJ/7ycqGkCYZwmAAmExZWtyRwnDQ6nCWJMqchJUm6aU+R2cpWAnKpLoefIWlCpoDAblurL6HuwiHkYx6WGe4KdHo77mLe+m/tcRgAS+4LXFeyVfp6D4JU+gFwO8wHBmBUFwQhMK0eCqamIxKX0axzHJfhgZEXAZGUTRnwoYlaF5ZhKG/jhcBNVxinCcZQkKCJrbOTpskFd6fkqVgvkmfNi3abJekXmt23yLN00KOZdZINZEC2cdC2WUt+1uXk61eacZx+btgVAA}{\underline{此处}}. 
        \begin{itemize}
            \item[P.S.] 笔者没想过其中的逻辑原因, 但还是莫名其妙地证出来了.
        \end{itemize} 
    \end{proof}
\end{example}

\begin{remark}
    问题出在哪儿? 
\end{remark}

\begin{exercise}
    给定自然数 $l$, $m$ 与 $n$, 证明 
    \begin{enumerate}
        \item $\max(\min(l,m),\min(l,n))=\min(l,\max(m,n))$, 
        \item $\min(\max(l,m),\max(l,n))=\max(l,\min(m,n))$.
    \end{enumerate}
    提示: $a = b$ 当且仅当 $(x\leq a)\leftrightarrow (x\leq b)$, 此时 $\max$ 对应``或'', $\min$ 对应``与'', ...
\end{exercise}

\begin{example}[min-max 不等式]
    (min-max 不等式) 给定任意实数 $a_1$, $b_1$, $a_2$ 与 $b_2$, 总有
    \begin{equation}
        \max ((\min (a_1, b_1)), (\min (a_2, b_2)) )≤ \min ((\max (a_1, a_2)), (\max (b_1, b_2))). 
    \end{equation}  
    \begin{proof}
        \href{https://live.lean-lang.org/#codez=JYWwDg9gTgLgBAWQIYwBYBtgCMB0ARFJHAJQFMl0cAhJAZ2AGMAoJtU6UkOEYAOwH0QSAB78+pAI4BXCsBgBPOAAokgQII4WdUkBBBBt0AuOIFxCAJRxDQ4cp684ajarNKbd3Vm1nAJkTc+1kXa0PPytNPTN9AF4NeSY4OzAwdEV0UkE+WLgceMTFS34UjLis1CQAN1I4dHMfW3tQ73tI6LjspJr81JSAMxhClqQEtpT+GCgkXlpKvriBnMrUvO7elsy4EvLK6pcdPTgGgyisRVm2mw6OnunW5NTR8cn0K5ObwRF+KGAAc1Rl1efuN4FFbFMoVKoWXx1dTeUJNI7XdrDD7fX4rf7DO4TKYrGaDF6LUiXYFrUGbCG1Ny6GEHZr/M5Ir4/J54+YjMZYx44hHDPLIplAA}{形式化证明}. 
    \end{proof}
\end{example}

\begin{remark}
    Min-max 不等式是极其广泛的: 假定 $f:X\times X\to P$ 是集合到偏序集合任意映射 (例如 $f:X\times Y\to \mathbb R$ 是二元实函数), 则恒有 
    \begin{equation}
        \sup_{x\in X}\inf_{y\in Y}f(x,y)\leq \inf_{y\in X}\sup_{x\in Y}f(x,y). 
    \end{equation}
    作为特例, 取 $X=Y=\mathbb N$, 以数列定义 $a_{m+n}=f(m,n)$, 则上式表示数学分析中何种的结论? 
\end{remark}

\begin{pinked}
    未完待续 (有些待补全的内容可以在上一届习题中找到). 
    \begin{enumerate}
        \item 目前引入 $\mathrm{span}$ 的方式还是很``牵强''的, 之后会有更自然的角度. 等课上提到了``泛性质''再话吧. 
        \item 此处未涉及线性空间的补空间. 
        \item 将``两个线性空间''换作``一族线性空间''? 
    \end{enumerate}
\end{pinked}















\newpage

\section{思考: 无限维线性空间}

\begin{exercise}[无限维的定义]
    $\mathbb F$ 上无限维线性空间的定义如下 (选自教材之一 \href{https://linear.axler.net/LADR4e.pdf#page=45}{LADR (第四版)}): 
    \vspace{-1cm}
    \begin{quotation}
        \begin{definition}
            称线性空间 $V$ 是无限维的, 若 $V$ 不是有限维的.
        \end{definition}
    \end{quotation}
    请证明: $V$ 是无限维的, 当且仅当存在无限集 $S\subset V$ 使得 $S$ 是线性无关组. 
    \begin{pinked}
        (若选择证明此题) 书写证明时, 换段地书写``充分性''与``必要性''是基本要求之一. 
        \begin{itemize}
            \item 以上命题存在不显然之处, 请在证明完毕后指出. 
        \end{itemize}
    \end{pinked}
\end{exercise}

\begin{remark}
    以上习题通常被默认作``常识''; 此处有一先决条件, 就是 $\mathrm{span}(S)$ 的定义. 
\end{remark}

\begin{exercise}[Challenging]
    如果你熟悉数学分析中的 Cantor 对角线法, 不妨尝试以下命题: 
    \begin{itemize}
        \item 形式幂级数空间 $\mathbb F[\![x]\!]$ 不是可数维线性空间; 
    \end{itemize}
    换言之, 对任意可数集 $S\subset \mathbb F[\![x]\!]$, 总有 $\mathrm{span}(S)\neq \mathbb F[\![x]\!]$. (以上 $\mathbb F$ 是任意域.)
\end{exercise}

\newpage 



\end{document}